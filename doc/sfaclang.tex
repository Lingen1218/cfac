\chapter{SFAC Language Reference}
\label{cha:function}

\section{Functions}
\label{sec:fac}

This section describes the functions available in the SFAC interface. In the
documentation of each function, the arguments in brackets are optional,
arguments separated by ``$\mid$'' are alternative forms of calling syntax,
``...'' in the argument list denotes variable number of arguments, and keyword
arguments are indicated by \var{key=arg} pair.

\subsection{Basic Configuration}

\begin{fundesc}{SetAtom}{asym\opt{, z\opt{, m\opt{,r}}}}
This function sets the atomic element to \var{asym}, where \var{asym} is a
standard chemical element symbol. The nuclear charge \var{z}, atomic mass
\var{m}, and the nucleus radius of the element can be set optionally. If they
are not set, the standard values are used.
\end{fundesc}

\begin{fundesc}{SetFields}{b, e, a\opt{, m}}
Set the magnetic and electric fields. \var{b} is the magnetic fields in gauss,
\var{e} is the electric fields in V/cm. \var{a} is the angle between the
magnetic and electric fields. If the optional \var{m} is 1, then the
diamagnetic effects are ignored in the Hamiltonian.
\end{fundesc}

\begin{fundesc}{SetUTA}{m}
Set the flag for configuration average models. If $m=1$, all calculations are
carried out in the configuration average approximation. The radiative transition
rates output contains three additional fields, the transition energy including
the UTA shift, the Gaussian standard deviation, and the correction to the line
strengths due to the configuration interaction within the same non-relativistic
configurations. If $m=0$, which is the default, the usual detailed term
accounting method is used. Normally, this function should be called in the
beginning of the script, before invoking \funcref{Config}. However, see
\hyperref[subsec:mixed_mode]{Mixed mode of calculations} below.

\subsection{Mixed Mode of Calculations (experimental)}
\label{subsec:mixed_mode}
Normally, the \funcref{SetUTA} function should be called in the beginning of a
script, before the \funcref{Config} call(s). However, since version 1.6.0, an
experimental {\em mixed} mode of calculations, allowing for mixing detailed and
UTA levels in the same calculations, is possible. To this end, one may call
\funcref{SetUTA} several times in between \funcref{Config} calls, with each
\funcref{Config} being treated in either UTA or non-UTA (detailed) mode
depending on the last preceeding \funcref{SetUTA} setting.

The other, preferred way of performing mixed-mode calculations is to pass the
UTA flag to the \funcref{Config} function itself.
\end{fundesc}

\subsection{Electron Shell Configuration}

\begin{fundesc}{Closed}{s, ...}
Specify the closed shells in the electronic configurations. It takes a variable
number of arguments, each of them is a non-relativistic or relativistic shell in
the spectroscopic notation. For example, \key{2s} for $2s$ shell, \key{2p-} for
$2p_{1/2}$ shell, and \key{2p+} for $2p_{3/2}$ shell.
\end{fundesc}

\begin{fundesc}{Config}{c, ..., group=g $\mid$ g, c, ... $\mid$ uta, g, c, ...}
Add one or more configurations to the configuration group \var{g}. In the first
form, the group name \var{g} is given as a keyword, while in the second form,
the first argument must be a group name instead of a configuration. It takes one
or more strings for the configuration specification. A configuration \var{c} is
a string comprised of one or more non-relativistic or relativistic shells in
spectroscopic notation separated by white spaces. For example, \key{2[p+]3} is a
$2p_{3/2}$ shell with 3 electrons. If an asterisk (\key{*}) is given instead of
the orbital angular momentum symbol, configurations with all legitimate values
are generated. It is also possible to use \key{[s,p,d]} to indicate that the
orbital angular momentum may take $s$, $p$, or $d$ values. For $l > 20$, no
spectroscopic symbols are defined, but it can be specified as [$l$], such as
[21,22] for the $l$ = 21 and 22 shells. This numerical notation also works for
$l \le 20$ shells, for which the spectroscopic symbols \key{s, p, d, f, g, h, i,
k, l, m, n, o, q, r, t, u, v, w, x, y, z} are available. The brackets for the
orbital angular momentum can be omitted if it consists of a single character.
Otherwise, the brackets must be present. For example, $2[p-]2$  is legal, but
$2p-2$ is not. Each shell may be followed by multiple conditions on the
occupation number(s), separated by ``;'', e.g., $3*10;3s>0;3p>5$ generate
configurations that have at least 1 electron in the $3s$ shell and at least 6
electrons in the $3p$ shell. The logical relations allowed in conditions
include $=$, $>$, and $<$.

The third form of invoking this function allows for running calculations in the
{\em experimental} \hyperref[subsec:mixed_mode]{mixed mode}, with the \var{uta}
flag explicitly specifying whether the given configuration should be treated in
the UTA (\var{uta} = 1) or detailed (\var{uta} = 0) way. By default (i.e., with
the \var{uta} flag omitted), the mode is set by the last preceeding call to
\funcref{SetUTA}.
\end{fundesc}

\begin{fundesc}{GetConfigNR}{c, ...}
This functions returns a list of non-relativistic configurations corresponding
to the supplied configuration strings, which may contain wild casts as in the
\funcref{Config} function. It is useful, e.g., when one wants to know the
non-relativistic configurations of \key{1*2 2*2 3*1}.
\end{fundesc}

\begin{fundesc}{ListConfig}{\opt{fn, g}}
Print the configurations in the list \var{g} to file \var{fn}. If \var{fn} is
not given or if it is ``-'', the results are written to the stdout. If \var{g}
is not given, then all configurations currently defined are printed.
\end{fundesc}


\subsection{Structure Calculations}

\begin{fundesc}{AvgConfig}{c}
Set up a mean configuration for the optimization of central potential as
specified by the string \var{c}. The format of the string is the same as in the
function \funcref{Config}, except that the occupation can be a non-integer
number in this routine. The mean configuration set up by this function is
effective only if the function \funcref{OptimizeRadial} is called with no
arguments. Otherwise, the configurations given in that function are used to
generate the mean configuration automatically. It is important that this
function be called before \funcref{OptimizeRadial} and after
\funcref[0]{ConfigEnergy}, if the latter is used.
\end{fundesc}

\begin{fundesc}{BasisTable}{fn\opt{,m}}
Print out a table of basis wavefunctions and mixing coefficients in the file
\var{fn}. If \var{m} is 0, then the basis table for the ordinary atom is
given, otherwise, the basis table for atom in magnetic and electric fields are
given.
\end{fundesc}

\begin{fundesc}{ConfigEnergy}{m\opt{, n\opt{, g, ...}}}
This function should be called twice just before (with \var{m} = 0) and after
(with \var{m} = 1) \funcref{OptimizeRadial} if used. If \funcref{AvgConfig} is
called, then \funcref{ConfigEnergy} with \var{m}=0 must be called before that.
The call with \var{m} = 0 performs a radial optimization for the configuration
groups given by the list \var{g}, and calculate the average energy of each
configuration under such potentials. Multiple optimizations are performed if
more than one list are given. If none is given, the optimization are carried out
for each configuration group. If \var{m} = 0, one may also specify an integer
\var{n} to indicate that \funcref[n]{RefineRadial} should be called after the
optimization. The call with \var{m} = 1 does not accept additional arguments. It
recalculates the average energy of each configuration under the potential
obtained by \funcref{OptimizeRadial} issued by the user. The diagonal elements
of the Hamiltonian calculated in the \funcref{Structure} call are then adjusted
by the difference of the two average energies for each configuration. The
purpose of this routine is to remove some of the errors in the level energies
introduced by using a single central potential for all configurations.
\end{fundesc}

\begin{fundesc}{CorrectEnergy}{fn, nmin $\mid$ ilev, e, nmin}
Correct the energies of certain levels by given amount. This should be used if
the exact energy of some levels is critical. In the first form, the indexes and
the energy corrections are listed as two columns in the file \var{fn}. In the
second form, the indexes are given in the list \var{ilev}, and the energy
corrections are given in the list \var{e}. Only levels with valence electron in
$n\ge nmin$ are corrected, if these levels are not constructed with
\funcref{RecStates}. The corrections are given in units of eV. For the first
level in the correction list, the amount is added to the energies calculated by
\cFAC, and that level is taken as the reference level for all other entries in
the list. The correction energies in the list \var{e} for these other entries
with \var{ilev}$\ge 0$ are the desired energies relative to the reference level,
i.e., they replace the energies calculated by \cFAC. However, if \var{ilev}$<0$,
then these corrections are also interpreted as shifts to be added to the
energies calculated by \cFAC.
\end{fundesc}

\begin{fundesc}{CutMixing}{g0, g1\opt{, c}}
For each levels in the group list \var{g0}, eliminate all mixing components
that are not in the group list \var{g1} or the mixing coefficients less than
\var{c}.
\end{fundesc}

\begin{fundesc}{GetPotential}{fn}
Print the radial potential obtained by \funcref{OptimizeRadial} to the file
\var{fn}. The file starts with the parameters for the analytic fit the to
potential, $\lambda$ and $a$, in the formula
\begin{equation}
V_0(r) = -\frac{Z}{r} + \frac{N-1}{r}\left(1-\frac{\exp(-\lambda
r)}{1+ar}\right).
\end{equation}
After that, the mean configuration used to generate the potential is printed
in 3 columns representing the principal quantum number, the relativistic
angular quantum number $\kappa$, and the fractional occupation number
respectively. Finally, the file gives 8 columns which are $i$, $r$, $Z(r)$,
$V(r)$, $V_d(r)$, $V_e(r)$, $V_e^\prime(r)$, and $U(r)$, where $i$ is the index
for the radial grid, $r$ is the radial grid, $Z(r)$ is the nuclear charge at
radius $r$ taking into account the nuclear charge distribution, $V(r)$ is the
optimal potential, $V_d(r)$ is the direct interaction part of the potential,
$V_e(r)$ is the exchange interaction part of the potential, $V_e^\prime(r)$ is
the Slater approximation of the exchange interaction, and $U(r)$ is the
Uehling potential which approximates the vacuum polarization effects.
\end{fundesc}

\begin{fundesc}{OptimizeRadial}{\opt{g\opt{, w}}}
Obtain the optimal radial potential based on the mean configuration generated
by the configuration group list \var{g} and the weight \var{w}, or if they are
absent in the call, by the mean configuration specified by \funcref{AvgConfig}.
\var{g} and \var{w} must be equal length lists if both present. \var{g}
is a list of configuration groups, and \var{w} is a list of weights for each
group when generating the mean configuration. If only \var{g} is present, each
configuration group is given an equal weight.
\end{fundesc}

\begin{fundesc}{PrepAngular}{p\opt{, q}}
Precalculate the angular coefficients between states in \var{p} and
\var{q}. \var{p} and \var{q} are lists of configuration groups. If \var{q} is
not present, the angular coefficients are calculated between states in the
\var{p} list. Only $Z^L(\alpha,\beta)$ and $\tilde{a_\alpha}$ coefficients are
calculated. If the bra and ket states have the same number of electrons, $Z^L$
is calculated, otherwise $\tilde{a_\alpha}$ is calculated. This routine should
primarily be used when atomic states are constructed with \funcref{RecStates},
where the angular coefficients between the base states are used many times. It
is therefore more efficient to precalculate these coefficients.
\end{fundesc}

\begin{fundesc}{RecStates}{fn, b, n}
Construct recombined states by adding a spectator electron with the principal
quantum number \var{n} onto the basis states in the configuration groups
\var{b}. The orbital angular momentum of the spectator electron is set by two
functions \funcref{SetRecPWLimits} and \funcref{SetRecSpectator}. The resulting
energy levels are saved in file \var{fn}.
\end{fundesc}

\begin{fundesc}{RefineRadial}{\opt{n\opt{, m}}}
This function may be called after \funcref{OptimizeRadial}, which performs a
minimization of the total energy of the mean configuration by adjusting the
parameters in the analytic central potential. \var{n} is the number of energy
evaluations allowed in the minimization, and \var{m} controls the print out
during the calculation. Default: \var{n} = 250, \var{m} = 0 (no print out).
\end{fundesc}

\begin{fundesc}{SetAngZCut}{c}
Set the cutoff threshold for the mixing basis in the calculation of recoupling
coefficients. Only the basis functions with mixing coefficients $>$\var{c} are
included. The default is $10^{-5}$ if this routine is not called.
\end{fundesc}

\begin{fundesc}{SetBreit}{n}
Set the maximum principal quantum number of the orbitals for which the
Breit interaction should be included in the Hamiltonian. If \var{n} $<$ 0,
then the Breit interaction involving all bound and continuum states is
included. The default value is 5.
\end{fundesc}

\begin{fundesc}{SetCILevel}{m}
Set the level of configuration interaction space. By default, $m=0$, the
configuration space is determined by the configuration groups passed to the
\funcref{Structure} function. CI can be further refined by the value of $m$. If
$m=-1$, then no CI is included, the Hamiltonian is assumed to be diagonal. If
$m=1$, only CI within the same relativistic configuration is included. If
$m=2$, only CI within the same non-relativistic configuration is included. If
$m=3$, only CI within the same configuration group is included.
\end{fundesc}

\begin{fundesc}{SetHydrogenicNL}{\opt{n,\opt{l}}}
Set the principal quantum number \var{n} and the orbital angular momentum
\var{l}, beyond which, the hydrogenic approximation for the E1 multipole
integrals should be used. If this routine is not called or the arguments are not
given, the default values of \var{n} = 20 and \var{l} = 10 are used.
\end{fundesc}

\begin{fundesc}{SetMaxRank}{k}
Set the maximum rank in the expansion of the Slater integrals. The default is
20, if this routine is not called.
\end{fundesc}

\begin{fundesc}{SetMixCut}{c}
Set cutoff threshold of the mixing basis in the wavefunction. Only the basis
eigenvectors with mixing coefficients greater than \var{c} are included in the
wavefunction expansion. Default is $10^{-5}$.
\end{fundesc}

\begin{fundesc}{SetMS}{nms, sms}
Set flags for the QED normal and specific mass shifts contributions to the
Hamiltonian. 0---disable, 1---enable. Defaults: 1.
\end{fundesc}

\begin{fundesc}{SetOptimizeControl}{t, s, m\opt{, p}}
Set the options for radial potential optimization. \var{t} is the tolerance for
the self-consistent field iteration. \var{s} is the stabilizer for the
iteration, a real number between 0 and 1. \var{m} is the maximum number of
iterations allowed. \var{p}  specifies whether diagnostic information should be
printed out during the optimization. This routine does not need to be called.
The default for \var{t} is $10^{-6}$, \var{s} is determined dynamically
according to the type of ion, \var{m} is 128, and \var{p} is 0 for no printing
out of information.
\end{fundesc}

\begin{fundesc}{SetOptimizeMaxIter}{m}
Set the maximum number of iterations allowed in \funcref{OptimizeRadial}. See
\funcref{SetOptimizeControl}.
\end{fundesc}

\begin{fundesc}{SetOptimizePrint}{m}
Set the printing option in \funcref{OptimizeRadial}. See
\funcref{SetOptimizeControl}.
\end{fundesc}

\begin{fundesc}{SetOptimizeStabilizer}{m}
Set the stabilizer factor in \funcref{OptimizeRadial}. See
\funcref{SetOptimizeControl}.
\end{fundesc}

\begin{fundesc}{SetOptimizeTolerance}{m}
Set the tolerance factor in \funcref{OptimizeRadial}. See
\funcref{SetOptimizeControl}.
\end{fundesc}

\begin{fundesc}{SetRadialGrid}{n\opt{, r0\opt{,r1}\opt{,rmin}}}
Set the radial grid properties. \var{n} is the number of radial grid points. It
must be an even number and less than the macro \key{MAXRP} (3000). \var{r0}
specifies the ratio of successive radial points near  origin, which is
approximately logarithmic. \var{r1} specifies the number of mesh-points per
oscillation wavelength for very high-$n$ orbitals at large  radii. \var{rmin}
divided by the nuclear charge is the starting point of the radial mesh.
\end{fundesc}

\begin{fundesc}{SetScreening}{ns\opt{,c\opt{,k}}}
Set the orbital parameters of screening electrons. \var{ns} is a list of
integers which are the principal quantum numbers of the screening
orbitals. The optional \var{c} is the total charge to be screened, whose
default is 1.0. \var{k} is either 1, $-1$, or 0. If \var{k}=$-1$, then
the \var{l=0} orbitals are used. If \var{k}=0, then the the \var{l}=\var{ns}/2
orbitals are used. If \var{k}=1, then the \var{l}=\var{ns}-1 nodeless orbital
is used. The default is \var{k}=1. This function is usually used when
additional screening charge is desired for the mean configuration generating
the optimal central potential. It is quite experimental, and therefore not
recommended for general use.
\end{fundesc}

\begin{fundesc}{SetSE}{n}
Set the maximum principal quantum number of the orbitals for which the QED
self-energy correction should be included in the Hamiltonian. Default is 5.
\end{fundesc}

\begin{fundesc}{SetSlaterCut}{k1, k2}
Set the calculation modes of Slater integrals. \var{k1} and \var{k2} are orbital
angular momentum values. When one of the orbitals has $l > k1$, then exchange
integrals are not calculated. When $l > k2$, the direct integrals are evaluated
with the multipole moments.
\end{fundesc}

\begin{fundesc}{SetSymmetry}{p, j}
Define which $\pi J$ symmetry to include in the structure calculations. By
default, all symmetries are processed. But if this function is called with $p=0$
or 1 and $j \ge 0$ or $j$ is a list of integers, then only the specified
symmetry is processed. $j$ (or integers in the list if $j$ is a list) is twice
the actual value of the total angular momentum.
\end{fundesc}

\begin{fundesc}{SetVP}{vp}
Set the flag for QED vacuum polarization correction to the Hamiltonian.
0---disable, 1---only include 2nd order term, 2---include the 4th order term as
well. Default is 2.
\end{fundesc}

\begin{fundesc}{SlaterCoeff}{fn, g, a, b}
Calculate the expansion coefficients of the exchange radial integral in the
Coulomb energy of each state in the configuration list  \var{g}. The Coulomb
energy between electrons of a state can be expanded as
\begin{equation}
E = \sum g_kG^k(\alpha\alpha^\prime,\beta\beta^\prime) + \mbox{direct terms},
\end{equation}
where $\alpha$ and $\alpha^\prime$ are the interacting orbitals in the bra and
ket states that have the same $l$ values. Because we work in the $jj$-coupling
basis, $\alpha$ and $\alpha^\prime$ may have different $j$ values. The results
are stored in the text file \var{fn}. These coefficients are useful in
determining the radiative transition rates from level to average configurations.
\var{a} and \var{b} are the orbital lists containing $\alpha$ and $\beta$,
respectively. They are specified as configuration strings. The format of the
file is as follows. For each state, a line beginning with ``\#'' gives the level
index, parity, $2J$-value, and the configuration label. It is followed by a
block of lines. In this block, the first column is the level index. The 2rd,
3th, and 4th columns are the $n$, $kappa$, and $l$ values of the $\alpha$
orbital, respectively. The 5th column is either 0 or 1, indicating whether
$\alpha$ and $\alpha^\prime$ have the same $j$ values. The 6--9th  columns are
the corresponding values for the $\beta$ orbital. The 10th column is the
expansion coefficients $g_1$. The 11th column is $g_2$. The 12th column is the
number to be added to $g_1$ to form the relative intensities of level to
configuration transitions for dipole transitions. The 13th column is the number
to be added to $g_2$ to form relative intensities for quadrupole transitions.
\end{fundesc}

\begin{fundesc}{Structure}{fn, g\opt{, p\opt{, ip}} $\mid$ p, j}
Diagonalize the Hamiltonian for configurations in the groups \var{g}. The
configurations in the optional groups \var{p} are allowed to interact with
\var{g} but only states within \var{g} are added to the energy level table. If
\var{ip}=0 (default), the interaction between \var{g} and \var{p} are treated
exactly, if \var{ip}=1, this interaction is treated approximately in a way that
the non-diagonal elements within \var{p} are neglected. The energy levels are
output to the file \var{fn}.

The second, obsolete form of this function corresponds to
\funcref[p, j]{SetSymmetry}.
\end{fundesc}

\begin{fundesc}{StructureEB}{fn, g}
Calculate the atomic structure for atom in magnetic and electric fields. The
levels belonging to the configuration group \var{g} are allowed to mix in the
external fields. This function should be called after \funcref{Structure}.
\end{fundesc}

\begin{fundesc}{WaveFuncTable}{fn, n, k\opt{, e}}
Print the radial wavefunction of the orbital with the principal quantum number
\var{n} and the relativistic angular quantum number $\kappa$=\var{k} to the file
\var{fn}. If \var{n}=0, the orbital is a continuum state. In this case, the
optional \var{e} must be a positive number for the energy of the continuum
orbital in units of eV.
\end{fundesc}


\subsection{Radiative Transitions}

\begin{fundesc}{SetTransitionGauge}{g}
Set the gauge for radiative transition. 1 for Coulomb gauge (velocity form) and
2 for Babushkin gauge (length form, which is the default).
\end{fundesc}

\begin{fundesc}{SetTransitionMaxE}{max\_e}
Set the maximum rank of electric multipole transitions. Default is 4.
\end{fundesc}

\begin{fundesc}{SetTransitionMaxM}{max\_m}
Set the maximum rank of magnetic multipole transitions. Default is 4.
\end{fundesc}

\begin{fundesc}{SetTransitionMode}{m}
Set the mode for the multipole integral calculation. 0 for fully relativistic
and 1 for non-relativistic approximation (the default is 1, but $M1$ transitions
are always calculated in the fully relativistic mode anyway).
\end{fundesc}

\begin{fundesc}{SetTransitionOptions}{g, m\opt{, max\_e, max\_m}}
A convenience function combining the four functions above in a single call.
\end{fundesc}

\begin{fundesc}{TransitionTable}{fn, low, up\opt{, m}}
Calculate the weighted oscillator strength and radiative transition rates from
states in configuration group list \var{up} to states in the configuration list
\var{low} with multipole type \var{m}. The default for \var{m} is -1, i.e., E1
transitions. The results are saved to file \var{fn}.
\end{fundesc}

\begin{fundesc}{TRTable}{fn, low, up\opt{, m}}
Same as \funcref{TransitionTable}
\end{fundesc}

\begin{fundesc}{TRTableEB}{fn, lo, up\opt{, m}}
Calculate radiative transition rates between levels in external magnetic and
electric fields.
\end{fundesc}


\subsection{Collisional Excitation}

\begin{fundesc}{CETable}{fn, low, up}
Calculate the collision strength for the excitation of states in the
configuration group list \var{low} to those in the group list\var{up}. The
results are saved in the file \var{fn}.
\end{fundesc}

\begin{fundesc}{CETableEB}{fn, low, up\opt{, m}}
Calculate the collision strength for the excitation of states in the
configuration group list \var{low} to those in the group list\var{up}, but for
atoms in magnetic and electric fields. The
results are saved in the file \var{fn}. If \var{m} is 0, then the incident
electron is assumed to be isotropic, otherwise, cross sections at different
incident directions are calculated.
\end{fundesc}

\begin{fundesc}{CETableMSub}{fn, low, up}
Calculate the magnetic sublevel collision strength for the excitation of states
in the configuration group list \var{low} to those in the configuration group
list \var{up}. The results are saved in the file \var{fn}.
\end{fundesc}

\begin{fundesc}{SetCEBorn}{e\opt{,x\opt{, x1}}}
\var{e} specifies the asymptotic energy where the plane-wave Born approximation
is expected to be valid, and is used to deduce the constant component of the
collision strength at high energies. If $e > 0$, it is in units of the
characteristic transition energy of the array. If $e < 0$, then its absolute
value is the energy in units of eV. $x$ and $x1$ set the energy boundary between
the distorted-wave Born approximation and plane-wave Born approximation for the
collisional excitation. If $x > 0$ and the scattered electron energy is less
than $xE_{b}$, the DW method is chosen ($E_{b}$ is the binding energy of the
electron being excited). If $x$ is negative, the switch between DW and PW occurs
when the estimated high partial-wave contributions becomes larger than $-x$
times the low partial-waves calculated explicitly. If $x=0$, then PW is always
used. The default for $x$ is $-0.5$, i.e., use DW when the estimated
high-partial-wave contribution represents no more than 50\% of the explicitly
calculated low-partial-wave sum. The contributions of high partial waves for
monopole and dipole transitions are calculated within the Coulomb--Bethe
approximation, their cutoff values are treated differently with the switch
\var{x1}, whose default is $-1.0$.
\end{fundesc}

\begin{fundesc}{SetCEGrid}{g $\mid$ n\opt{, e0, e1}}
Set the collision energy grid for collisional excitation. In the first form, the
grid is given by a list \var{g}. In the second form, the grid is constructed
with \var{n} points, from \var{e0} to \var{e1}. The energies are specified for
the scattered electron energy, and are in units of eV. This routine does not
need to be called. A default grid is constructed for a given transition array
with 6 points, minimum and maximum energies specified by
\funcref{SetCEGridLimits}. Calling this routine with \var{n}=0, resets the grid
to the default one.
\end{fundesc}

\begin{fundesc}{SetCEGridLimits}{e0, e1}
Set the minimum and maximum collision energy for collisional excitation for the
automatic construction of the grid. \var{e0} and \var{e1} are in the units of
the average threshold energy of the transition array being considered. The
default ones are 0.05 and 8.0, respectively, if this routine is not called.
\end{fundesc}

\begin{fundesc}{SetCELCB}{m}
Set the maximal orbital angular momentum for Coulomb-Bethe approximation.
Default is 36.
\end{fundesc}

\begin{fundesc}{SetCELMax}{m}
Set the maximum of orbital angular momentum for the partial-wave expansion in
collisional excitation. Default is 36.
\end{fundesc}

\begin{fundesc}{SetCELQR}{m}
Set the maximal orbital angular momentum for quasi-relativistic approximation
in collisional excitation. The default is 0, i.e., always use the
quasi-relativistic approximation.
\end{fundesc}

\begin{fundesc}{SetCEQkMode}{mode}
Set the computation mode for the excitation radial integrals. \var{mode} may
be a string or an integer specifying the mode. These values are listed in the
variable \key{QKMODE}.
\end{fundesc}

\begin{fundesc}{SetTEGrid}{g $\mid$ n\opt{, e0, e1}}
Set the transition energy grid for collisional excitation. In the first form,
the grid is given by a list \var{g}. In the second form, the grid is constructed
with \var{n} points from  \var{e0} to \var{e1}. This routine does not need to be
called. A 3-point grid is constructed by default. Calling this function with
\var{n}=0 resets the grid to the system default.
\end{fundesc}

\begin{fundesc}{SetUsrCEGrid}{g $\mid$ n\opt{, e0, e1}}
Set the user collision energy grid for collisional excitation. The collision
strengths on this grid are output. It is forced to be same as that set by
\funcref{SetCEGrid} if \key{QKMODE} is \key{'exact'}.
\end{fundesc}



\subsection{Collisional Ionization}

\begin{fundesc}{CITable}{fn, b, f}
Calculate the collision strength for the ionization of states in the bound
configuration groups \var{b} to those in the free configuration groups
\var{f}. The results are saved in the file \var{fn}.
\end{fundesc}

\begin{fundesc}{CITableMSub}{fn, b, f}
Calculate the magnetic sublevel collision strength for the ionization of
states in the bound configuration groups \var{b} to those in the free
configuration groups \var{f}. The results are saved in the file \var{fn}.
\end{fundesc}

\begin{fundesc}{SetCIEGrid}{g $\mid$ n\opt{, e0, e1}}
Set the collision energy grid for collisional ionization. In the first form, the
grid is given by a list \var{g}. In the second form, the grid is constructed
with \var{n} points, from \var{e0} to \var{e1}. The energies are specified for
the total energy of the final electrons, and are in units of eV. This routine
does not need to be called. A default grid is constructed for a given transition
array with 6 points, with the minimum and maximum energies specified by
\funcref{SetCIEGridLimits}. Calling this routine with \var{n}=0 resets the grid
to the system default.
\end{fundesc}

\begin{fundesc}{SetCIEGridLimits}{e0, e1}
Set the minimum and maximum collision energy for collisional ionization for
the automatic construction of the grid. They are in units of average threshold
energy of the transition array being considered. The default is 0.05 and 8.0
if this routine is not called.
\end{fundesc}

\begin{fundesc}{SetCILCB}{m}
Set the maximum orbital angular momentum for the Coulomb-Bethe approximation in
DW collisional ionization. Default is 36.
\end{fundesc}

\begin{fundesc}{SetCILMax}{m}
Set the maximum orbital angular momentum for the partial-wave expansion in DW
collisional ionization. Default is 36.
\end{fundesc}

\begin{fundesc}{SetCILMaxEject}{m}
Set the maximum orbital angular momentum for the ejected electron in DW
collisional ionization. Default is 4.
\end{fundesc}

\begin{fundesc}{SetCILQR}{m}
Set the orbital angular momentum for quasi-relativistic approximation in DW
collisional ionization. The default is 0, i.e., always use the
quasi-relativistic approximation.
\end{fundesc}

\begin{fundesc}{SetCITol}{m}
Set the tolerance factor in the partial-wave expansion of DW collisional
ionization. Default: 0.05.
\end{fundesc}

\begin{fundesc}{SetCIQkMode}{mode}
Set the computation mode for the ionization radial integrals. \var{mode} may
be a string or an integer specifying the mode. These values are listed in the
variable \key{QKMODE}.
\end{fundesc}

\begin{fundesc}{SetIEGrid}{g $\mid$ n\opt{, e0, e1}}
Set the ionization threshold energy grid for the collisional ionization. In
the first form, the grid is given by a list \var{g}. In the second
form, the grid is constructed with \var{n} points from \var{e0} to
\var{e1}. This routine does not need to be called. A 3 point grid is
constructed according to the transition array being considered by default.
Calling this routine with \var{n}=0 resets the grid to the system default.
\end{fundesc}

\begin{fundesc}{SetUsrCIEGrid}{g $\mid$ n\opt{, e0, e1}}
Set the user collision energy grid for collisional ionization. The collision
strengths on this grid are output.
\end{fundesc}



\subsection{Radiative Recombination and Photoionization}

\begin{fundesc}{RRMultipole}{fn, b, f\opt{, m}}
Calculate the bound-free multipole matrix elements. Because \funcref{RRTable}
does not calculate the matrix elements directly, this function exists to provide
such data. The output is written to the file \var{fn} in an ASCII format,
described in the header of the file itself. The remaining arguments are
identical to those of \funcref{RRTable}.
\end{fundesc}

\begin{fundesc}{RRTable}{fn, b, f\opt{, m}}
Calculate the bound-free differential oscillator strengths between the bound
configuration groups \var{b} and the free groups \var{f}, which are related to
radiative recombination and photoionization cross sections. The optional
multipole type \var{m} is set to -1 (E1) by default. In almost all cases, no
other multipole types should be important. The results are saved in file
\var{fn}.
\end{fundesc}


\begin{fundesc}{SetPEGrid}{g $\mid$ n\opt{, e0, e1}}
Set the free electron energy grid for photoionization, radiative recombination
and autoionization. In the first form, the grid is given by a list
\var{g}. In the second form, the grid is constructed with \var{n} points from
\var{e0} to \var{e1}. The energies are in units of eV. This function does not
need to be called. A 6 point grid is constructed according to the transition
array being considered by default. Calling this function with \var{n}=0 reset
the grid to system default.
\end{fundesc}

\begin{fundesc}{SetPEGridLimits}{e0, e1}
Set the minimum and maximum collision energy for photoionization and radiative
recombination for the automatic construction of the grid. They are in units of
average threshold energy of the transition array being considered. The default
is 0.05 and 8.0 if this routine is not called.
\end{fundesc}

\begin{fundesc}{SetRRTEGrid}{g $\mid$ n\opt{, e0, e1}}
Set the transition energy grid for photoionization and radiative recombination.
In the first form, the grid is given by a list \var{g}. In the second form, the
grid is constructed with \var{n} points from \var{e0} to \var{e1}. This routine
does not need to be called. For $E1$ type transitions, the transition energy
does not enter the calculation, a 1-point grid is constructed by default. for
other types of multipoles, a 3-point grid is constructed. Calling this function
with \var{n}=0 resets the grid to system default.
\end{fundesc}

\begin{fundesc}{SetRecPWLimits}{l0, l1}
Set the orbital angular momentum range for the spectator electron in the
recombined states to [\var{l0}, \var{l1}] inclusive. The default is [0, 12].
\end{fundesc}

\begin{fundesc}{SetRecPWOptions}{lmax}
Set maximum orbital angular momentum of the spectator electron in the
recombined states to \var{lmax}. The allowed values are also limited by the
setting of \funcref{SetRecPWLimits}. The default is 12.
\end{fundesc}

\begin{fundesc}{SetRecQkMode}{mode}
Set the computation mode for the photoionization and radiative recombination
radial integrals. \var{mode} may be a string or an integer specifying the
mode. These values are listed in the variable \key{QKMODE}.
\end{fundesc}

\begin{fundesc}{SetRecSpectator}{nmin\opt{, nfrozen}}
Set the minimum principal quantum numbers \var{nmin} and \var{nfrozen} for the
spectator electron. States with $n > nmin$ are constructed with
\funcref{RecStates} function, and those with $n > nfrozen$ are treated with the
frozen core approximation. Default for both \var{nmin} and \var{nfrozen} are 8.
\end{fundesc}

\begin{fundesc}{SetUsrPEGrid}{g $\mid$ n\opt{, e0, e1}}
Set the user electron energy grid for photoionization and radiative
recombination. The bound-free differential oscillator strengths are output on
this grid. It is forced to be same as that set by \funcref{SetPEGrid} if
\key{QKMODE} is \key{'exact'}.
\end{fundesc}


\subsection{Autoionization}

\begin{fundesc}{AITable}{fn, b, f}
Calculate the autoionization rates between the bound configuration group
\var{b} and the free configuration group \var{f}. The results are saved in
file \var{fn}.
\end{fundesc}

\begin{fundesc}{AITableMSub}{fn, b, f}
Calculate the magnetic sublevel autoionization rates and dielectronic capture
strength between the bound configuration group \var{b} and the free
configuration group \var{f}. The results are saved in file \var{fn}.
\end{fundesc}

\begin{fundesc}{SetAICut}{c}
Set the autoionization rate cutoff threshold in the output. Only
autoionization rates greater than \var{c} a.u. are output. The default is 0.
\end{fundesc}


\subsection{Data Storage and Manipulation}

\begin{fundesc}{AppendTable}{fn}
By default, when a new script is executed, existing binary files are
overwritten. If instead the new data should be appended to a file, use
this function to set the append flag.
\end{fundesc}

\begin{fundesc}{CheckEndian}{\opt{fn}}
Check the byte order of database file \var{fn}. It returns 0 for little endian
and 1 for big endian. If the optional file name \var{fn} is omitted, the
endian for the current platform is returned.
\end{fundesc}

\begin{fundesc}{JoinTable}{fn1, fn2, fn}
Join two binary files \var{fn1} and \var{fn2} to produce a single file
\var{fn}. \var{fn1} and \var{fn2} must have been produced on the same
platform, have the same type and be calculated for the same element.
\end{fundesc}

\begin{fundesc}{MemENTable}{fn}
Build an energy level table in the memory for the \texttt{DB\_EN} type file
\var{fn}. This function must be called before any calls to \funcref{PrintTable}
made in the verbose mode.
\end{fundesc}

\begin{fundesc}{PrintTable}{fnb, fna\opt{, v}}
Convert the binary database file \var{fnb} to the ASCII file \var{fna}. The
optional argument \var{v} = 1 requires the conversion be done in verbose
mode, otherwise it is done in simple mode. Note that before conversion in
verbose mode is carried out, one must call \funcref{MemENTable} first.
\end{fundesc}

\begin{fundesc}{StoreClose}{}
Close the database that has previously been initialized with
\funcref{StoreInit}.
\end{fundesc}

\begin{fundesc}{StoreInit}{fn \opt{, reset}}
Initialize a new session in the SQLite database file \var{fn}. If the file does
not exist, it will be created. If the optional flag \var{reset} = 1 is present
and the database already exists, all existing sessions will be dropped.
\end{fundesc}

\begin{fundesc}{StoreTable}{fn}
Store data contained in the binary file \var{fn} into a database that has
previously been initialized with \funcref{StoreInit}.
\end{fundesc}


\subsection{Miscellaneous}

\begin{fundesc}{Exit}{}
Exit SFAC.
\end{fundesc}

\begin{fundesc}{Info}{}
Print out the version information of \cFAC.
\end{fundesc}

\begin{fundesc}{Pause}{}
Pause execution of SFAC.
\end{fundesc}

\begin{fundesc}{Print}{arg1\opt{, ...}}
Print out the string representation of \var{args}.
\end{fundesc}


\section{Variables}
One can use variables instead of explicitly typed arguments of any SFAC
function. This can be convenient if the same argument is used in more than one
place. A variable name may contain any alphanumeric characters but must begin
with a letter. Function names listed in Sec.~\ref{sec:fac} cannot be used as
variables names for obvious reasons. Once assigned, a variable is accessed by
prepending the dollar sign (\$) to its name. For example,
\begin{verbatim}
a = 2
Print($a)
2
b = 'Hello'
Print($b)
Hello
\end{verbatim}
A variable may be either a scalar or a list (one-dimensional array), e.g.,
\begin{verbatim}
vv = [1,2,3]
Print($vv)
[1,2,3]
\end{verbatim}
