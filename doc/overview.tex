\chapter{Overview}
\label{cha:overview}

\section{What Are FAC and \cFAC}
FAC stands for the Flexible Atomic Code. It is an integrated software package to
calculate various atomic radiative and collisional processes, including energy
levels, radiative transition rates, collisional excitation and  ionization by
electron impact, photoionization, autoionization, radiative recombination and
dielectronic capture. FAC was written by Ming Feng Gu largely while at the Space
Science Laboratory of Berkeley. 

FAC makes detailed atomic model accessible to a wide community of laboratory and
astrophysical plasma diagnostics. Its flexible interface is designed to be
useful even for people without a deep understanding of the underlying atomic
theories. It is also powerful enough for experienced users to explore the
effects of algorithmic choices and different physical approximations.

The atomic structure calculation in FAC is based on the relativistic
configuration interaction with independent particle basis wavefunctions. These
basis wavefunctions are derived from a local central potential, which is
self-consistently determined to represent electronic screening of the nuclear
potential. Relativistic effects are fully taken into account using the Dirac
Coulomb Hamiltonian. Higher order QED effects are included with Breit
interaction in the zero energy limit for the exchanged photon, and hydrogenic
approximations for self-energy and vacuum polarization effects. Continuum
processes are treated in the distorted-wave (DW) approximation. Systematic
application of the factorization-interpolation method
makes the present code highly efficient for large scale calculations. The
details of theoretical background and computational methods are presented in
Chapter~\ref{ch:theory} of this manual.

\cFAC is a forked version of FAC. It was started by Evgeny Stambulchik
around 2010 (based on FAC--1.1.1, released in 2006), initially focusing on
providing large volumes of data as required, e.g., for collisional-radiative
plasma modeling, and replacing third-party Fortran numerical libraries with
their C equivalents (hence the change in the package name).

\section{License}
The bulk of the \cFAC sources is distributed under the GNU General Public
License (GPL) terms. The full text of the license is given in
\hyperref[GPL]{Appendix}.

Some bits of the FAC sources, which are still used in cFAC, were originally
published in the Computer Physics Communications (CPC) journal, and as such, are
licensed for non-profit or academic use only, please see
\url{http://cpc.cs.qub.ac.uk/licence/licence.html}.

In order to compile in these CPC-licensed modules, you need to pass the
``--with-cpc-module'' configure flag and explicitly agree to the CPC licensing
terms. {\em Please note that since CPC and GPL are incompatible, as a result,
the ``sfac'' executable will not be redistributable!}

\section{Obtaining and Installing \cFAC}
\index{Install}
\label{sec:install}
The latest version of \cFAC can be obtained from
\url{https://www-amdis.iaea.org/FAC/}. The public repository is hosted on
\href{https://github.com}{GitHub}, \url{https://github.com/fnevgeny/cfac}.

Much of the \cFAC package is written in ANSI C and Fortran 77. It should 
therefore work on any platform with modern C and Fortran 77 compilers.
\verb|gcc| and \verb|gfortran| from the
\href{https://gcc.gnu.org/}{GNU Compiler Collection} are recommended.

Step-by-step instructions for installation can be found in the README file in
the top directory of the \cFAC distribution.

\section{Quick Start}
\label{sec:start}
\subsection{SFAC Interface}
To use SFAC, one passes the
input files to the \verb|sfac| executable on
the command line such as 
\begin{verbatim}
    sfac input.sf
\end{verbatim}
or, one may invoke \verb|sfac| without arguments, in which
case, it reads from \verb|stdin| for inputs, where commands are interpreted
line by line.

Perhaps the quickest way to get familiar with \cFAC is to inspect the simple
demo scripts in the \verb|demo/| directory in \cFAC distribution. There
are individual scripts demonstrating the
calculation of energy levels, radiative transition rates, collisional
excitation and ionization cross sections, radiative recombination cross
sections and autoionization rates.

In this section, we look into the details of one of these scripts,
\verb|demo/structure/test.sf| for the calculation of Ne-like iron
energy levels and radiative transition rates between $n = 2$ and $n = 3$
complexes. The following is a duplication of that script.

\lstset{numbers=left,caption=demo/structure/test.sf}
\includecode[sh]{../demo/structure/test.sf}

Line numbers are added for easy reference, they are not part of the script. 
Empty lines are ignored. As is evident from the above listing, all functions
have a naming convention of concatenated capitalized words.
Line 3 set the atomic element to be iron. Line 5 is a comment, which starts
with a \verb|#|. Lines 6--11 specify the electronic configurations to be
included in the calculation. The closed shells specified by the function
\verb|Closed| must be inactive in this calculation. In the \verb|Config|
functions, \verb|2*8| stands for an $n = 2$ complexes with 8 electrons, while
\verb|2*7 3*1| stands for all configurations resulting from excitation of one
electron from $n = 2$ to $n = 3$. For more possibilities in the specification
of electronic configurations, one is referred to Chapter \ref{cha:function}.
Lines 14--21 carry out a Dirac-Fock-Slater self-consistent calculation to
derive a local central potential which represents the electronic screening of
the nuclear potential. In this calculation, the potential is optimized to the
average electron clouds of configurations \verb|n2| and \verb|n3|, since in
\cFAC, all atomic processes are treated with basis wavefunctions generated from
a single potential. This
results in the potential to be less optimized for \verb|n2| and \verb|n3|
individually. Lines 14 and 21 are used to make a crude correction to the
resulting energy levels due to this effect. The first call to
\verb|ConfigEnergy(0)| will make individual optimization to all configuration
groups. The average energy of each configuration group with these individually 
optimized potential is then calculated and stored. The
second call to \verb|ConfigEnergy(1)| will then recalculate the average energy
of configuration groups under the potential taking into account all
configuration groups. The difference between the two represents the effect of
a less optimized potential, and are used to adjust the final energy levels. If
this procedure is not needed, one can omit lines 14 and 21 in this script. Line
23 sets up the Hamiltonian matrix for levels in $n = 2$ and $n = 3$ complexes,
diagonalize it, and saves to the energy level information in the binary file
\verb|lev.bin|. Line 24 builds an in-memory table of energy levels, which is
used to convert the binary files to their ASCII counterparts in verbose mode,
such as done in Line 25, which converts \verb|lev.bin| to \verb|lev.asc| (the
last argument to \verb|PrintTable| indicates it be done in verbose mode). For
the conversion in simple mode (the last argument is 0), the in-memory table is
not needed, and Line 24 may be omitted. For the difference between the verbose
and simple ASCII files, see Chapter \ref{cha:format}. Line 27 calculates the
E1 oscillator strength and transition rates between configuration groups
\verb|n2| and \verb|n3|, and saves the results in the binary file
\verb|tr.asc|. The function \verb|TransitionTable| accepts an optional 4th
integer argument specifying the transition type. A negative integer means
electric multipole and a positive integer for magnetic multipole. The absolute
value of the integer indicates the rank of the multipole. Therefore, $-1$ would
be E1, $+1$ would be M1, etc. Without this argument, the default is E1, as is
done here. Line 28 converts the binary output to an ASCII file in verbose
mode. The exact formats of binary and ASCII files are explained in Chapter
\ref{cha:format}. Here we list the two ASCII files \verb|lev.asc| and
\verb|tr.asc| resulted from this calculation. 

\lstset{numbers=none,basicstyle=\scriptsize,caption=demo/structure/lev.asc}
\includecode[]{../demo/structure/ref/lev.asc}

\lstset{numbers=none,basicstyle=\normalsize,caption=demo/structure/tr.asc}
\includecode[]{../demo/structure/ref/tr.asc}

In file \verb|lev.asc|, the energy, parity, $2J$ ($J$ is the total
angular momentum of the level), and configuration coupling information are
listed. In file \verb|tr.asc|, the upper and lower level indexes, the $2J$
values of these levels, the transition energy, $gf$-values, radiative
decay rates, and the reduced multipole matrix elements are given.

\section{Acknowledgments}
Throughout the development of FAC, the discussion with Ehud Behar, Masao
Sako, Peter Beiersdorfer, Ali Kinkhabwala and Steven Kahn has been very
useful. Some Fortran 77 subroutines were retrieved from Computer Physics
Communications Program Library at
\url{http://www.cpc.cs.qub.ac.uk}.

The original development of FAC (during Dec 2000 -- Aug 2003, or prior
to version 1.0.2) was
supported by NASA through Chandra Postdoctoral Fellowship Award Number
PF01-10014 issued by the Chandra X-ray Observatory Center, which is operated
by Smithsonian Astrophysical Observatory for and on behalf of NASA under
contract NAS8-39073. 

Any opinions, findings and conclusions or
recommendations expressed in this manual are those of the author and do not
necessarily reflect the views of the National Aeronautics Space
Administration and/or the Smithsonian Astrophysical Observatory.
