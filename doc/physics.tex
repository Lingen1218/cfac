\chapter{Theory and Numerical Techniques}
\label{ch:theory}

This chapter is an extended version of theory and numerical techniques described
in Ref.~\cite{gu:2008a} and is based on a series of manuscript drafts
distributed with previous versions of FAC.

\section{Introduction}

A fully relativistic approach based on
the Dirac equation is used throughout the entire package which allows its
application to ions with large values of nuclear charge. Currently, \cFAC
is able to treat radiative transition, direct collisional excitation and
ionization by electron impact, non-resonant photoionization and radiative
recombination, autoionization and dielectronic recombination. These processes
are essential for the interpretation of laboratory and astrophysical 
spectroscopic data. The main goal of creating such a comprehensive package is
to integrate various atomic processes within a single theoretical framework,
ensure the self-consistency between different parts, and provide a uniform,
flexible and easy-to-use user interface for accessing all computational tasks. 

Many computer programs now exist for the calculation of atomic processes,
using either non-relativistic approximations (some including relativistic
effects through the Breit-Pauli Hamiltonian) or fully relativistic
methods. Most of them are mainly concerned with the atomic structure part and
bound-bound processes, e.g., the non-relativistic configuration interaction
codes CIV3 \cite{hibbert75} and SUPERSTRUCTURE \cite{eissner74}, the widely
used program of \citet{cowan81}, the multi-configuration Hartree-Fock (MCHF)
program of \citet{fischer00}, and the multi-configuration Dirac-Fock (MCDF)
code of \citet{grant80}. Many newer programs for continuum processes make
use of the output from these structure codes for bound state wave
functions. This sometimes leads to 
a different treatment of continuum states from that of bound states. More
importantly, the communication between different programs tends to complicate
the user interface, and makes it difficult for people other than the authors
of the codes to efficiently use them. 

There also exist several integrated
packages for the calculation of a variety of atomic processes, e.g., the ATOM
package \cite{amusia97}, the HULLAC package \cite{barshalom01}, and the fully
relativistic code (SZ) developed by \citet{sampson89} and
\citet{zhang89}. These programs treat continuum as well as bound
processes. In  
ATOM, the radial wave functions for bound and continuum orbitals are obtained
using the self-consistent field Hartree-Fock method or the frozen-core
Hartree-Fock method. A relativistic version based on the Dirac equation is
also available. Such a procedure tends to be very time consuming, especially
for continuum processes, where the number of continuum orbitals needed is
large. The HULLAC package and SZ are both based on the Dirac equation, and use
a single, local, central potential for the solution of radial orbitals. This
approach is very efficient because the orthogonality of 
different orbitals is automatically ensured. Both codes use the distorted wave
(DW) approximation for continuum processes. The difference between them is
mainly in the way the local central potential is obtained. In HULLAC, 
a parametric potential is used and the parameters in the potential are derived
by minimizing the average energies of some selected configurations
\cite{klapisch77}. In SZ, one 
constructs a fictitious mean configuration with fractional occupation numbers
that takes into account the electron screening of all configurations involved
in the physical processes to be calculated. A self-consistent
Dirac-Fock-Slater iteration is then 
performed on this mean configuration to derive the local central potential. 
Although various results from these two codes have been published over the
years, the programs are not available to the general public. The present
package combines the strengths of these existing atomic codes, with
modifications to numerical procedures made to extend the capability and
improve the efficiency and robustness.

\section{Structure}
\label{sec:structure}
The energy levels of an atomic ion with $N$ electrons are obtained by
diagonalizing the relativistic Hamiltonian $H$. In atomic units, which we shall
use throughout this chapter, it reads
\begin{equation}
\label{eq_hamilton}
H = \sum_{i=1}^{N} H_{D}(i) + \sum_{i<j}^{N}\frac{1}{r_{ij}},
\end{equation}
where $H_{D}(i)$ is the single-electron Dirac Hamiltonian for the potential
due to the nuclear charge. The basis states $\Phi_{\nu}$, which are usually
referred to as configuration state functions (CSF), are antisymmetric sums of
products of $N$ one-electron Dirac spinors $\varphi_{n\kappa m}$
\begin{equation}
\label{eq_spinor}
\varphi_{n\kappa m} = \frac{1}{r}\left(\begin{array}{c}
P_{n\kappa}(r) \chi_{\kappa m}(\theta, \phi, \sigma)\\
iQ_{n\kappa}(r) \chi_{-\kappa m}(\theta, \phi, \sigma)
\end{array}\right),
\end{equation}
where $\chi_{\kappa m}$ is the usual spin-angular function. $n$ is the
principal quantum number, $\kappa$ is the relativistic angular quantum number
, which is related to the orbital and total angular momentum through
\begin{equation}
\kappa = (l-j)(2j+1),
\end{equation}
and $m$ is the $z$-component of the total angular momentum $\mathbf{j}$. In
coupling the angular momenta of successive shells, the standard $jj$ coupling
scheme is used. 

The approximate atomic state functions are given by mixing the basis
states $\Phi_{\nu}$ with same symmetries
\begin{equation}
\label{eq_asf}
\psi = \sum_{\nu} b_{\nu} \Phi_{\nu},
\end{equation}
where $b_{\nu}$ are the mixing coefficients obtained from diagonalizing the
total Hamiltonian. 

\subsection{Choice of Local Central Potential}
The one-electron radial orbitals must be known in order to construct the
Hamiltonian matrix. In the standard Dirac-Fock-Slater method, the large and
small components, $P_{n\kappa}$ and $Q_{n\kappa}$, satisfy the coupled Dirac
equation for a local central field $V(r)$
\begin{eqnarray}
\label{eq_dirac}
\left(\frac{d}{d r} + \frac{\kappa}{r}\right)P_{n\kappa} &=&
\alpha\left(\varepsilon_{n\kappa} - V + \frac{2}{\alpha^2}\right)Q_{n\kappa}
\nonumber\\*
\left(\frac{d}{d r} - \frac{\kappa}{r}\right)Q_{n\kappa} &=&
\alpha\left(-\varepsilon_{n\kappa} + V\right)P_{n\kappa},
\end{eqnarray}
where $\alpha$ is the fine structure constant, and $\varepsilon_{n\kappa}$ are
the energy eigenvalues of the radial orbitals.

The local central potential $V$ includes the contribution from the nuclear
charge $V^N(r)$ and the electron-electron interaction $V^{ee}(r)$. The nuclear
potential can be written as
\begin{equation}
\label{eq_nuclear}
V^{N} = \Bigg\{\begin{array}{ll}
\frac{Z}{2}\left(\frac{r}{R_N}\right)
\left[3-\left(\frac{r}{R_N}\right)^2\right], & r \le R_N \nonumber\\*
Z, & r > R_N
\end{array},
\end{equation}
where $R_N$ is the statistical model radius of the nucleus, which can be
expressed in terms of the atomic mass $A$, $R_N = 2.2677\times 10^{-5}
A^{1/3}$ \cite{chernysheva99}. In the standard Dirac-Fock-Slater method, which
is the approach used by SZ, the electron-electron interaction includes the
spherically averaged 
classical potential due to the bound electrons, and a local approximation to
the exchange interaction
\begin{eqnarray}
\label{eq_ee}
V^{ee}(r) &=& V_c(r) - \left[\frac{3}{4\pi^2 r^2}\sum_{n\kappa}
\omega_{n\kappa}\rho_{n\kappa}(r)\right]^{1/3}, \nonumber\\*
V_c(r) &=& \sum_{n\kappa}\int \frac{\omega_{n\kappa}}{r_>}
\rho_{n\kappa}(r^\prime)d r^{\prime}, \nonumber\\*
\rho_{n\kappa}(r) &=& P_{n\kappa}^2(r) + Q_{n\kappa}^2(r), 
\end{eqnarray}
where $\omega_{n\kappa}$ is the occupation number of the subshell
$n\kappa$, and $r_>$ is the greater of $r$ and $r^{\prime}$. This potential
includes the undesirable self-interaction and has incorrect asymptotic
behavior. We shall use a slightly more complicated expression for $V^{ee}(r)$
\begin{eqnarray}
\label{eq_nee}
V^{ee}(r) &=& \frac{1}{r\sum_a
\omega_a\rho_a(r)}\Big\{\sum_{ab}\omega_a(\omega_b -
\delta_{ab})Y^0_{bb}(r)\rho_a(r) \nonumber\\*
&&+\sum_a\omega_a(\omega_a-1)\sum_{k>0}f_{k}(a,a)Y^{k}_{aa}(r)\rho_a(r) 
\nonumber\\*
&&+\sum_{a\ne b} \sum_k \omega_a\omega_b 
g_{k}(a,b) Y^{k}_{ab}(r)\rho_{ab}(r)\Big\},
\end{eqnarray}
where $a = n\kappa$ and $b = n^\prime\kappa^\prime$ are the dummy indices
denoting the subshells and 
\begin{eqnarray}
\label{eq_yk} 
\rho_{ab} &=& P_a(r)P_b(r) + Q_a(r)Q_b(r) \nonumber\\*
Y^k_{ab}(r) &=& r\int\frac{r_<^k}{r_>^{k+1}}\rho_{ab}(r^\prime)d r^\prime,
\end{eqnarray}
where $r_<$ and $r_>$ are the lesser and greater of $r$ and $r^\prime$,
respectively. $f_k$ and $g_k$ are the direct and exchange coefficients defined
as
\begin{eqnarray}
\label{eq_fg}
f_k(a,b) &=& -\left(1+\frac{1}{2j_a}\right)
\threej{j_a}{k}{j_b}{-\frac{1}{2}}{0}{\frac{1}{2}}^2 \nonumber\\*
g_k(a,b) &=& -\threej{j_a}{k}{j_b}{-\frac{1}{2}}{0}{\frac{1}{2}}^2,
\end{eqnarray}
where \threej{j_1}{j_2}{j_3}{m_1}{m_2}{m_3} is the Wigner $3j$ symbol. Such a
choice for the electron-electron interaction is based on the fact that the
quantity 
\begin{eqnarray}
\label{eq_Eee}
E^{ee} &=& \frac{1}{2}\sum_a\omega_a<a|V^{ee}|a> \nonumber\\*
&=& \frac{1}{2}\sum_a\omega_a\int V^{ee}(r)\rho_a(r)d r
\end{eqnarray}
is the electron-electron contribution to the average energy of the
configuration. The factor $1/2$ in Eq.~(\ref{eq_Eee}) accounts for the
double counting of electron pairs in the summation. It is easily seen
that Eq.~(\ref{eq_nee}) has the correct asymptotic behavior at large $r$,
since the self-interaction term is explicitly excluded.

In order to take into account the screening of more than one configuration
involved in the physical processes to be calculated, we do not use a single
physical configuration in the construction of the potential. Instead, a
fictitious mean configuration with fractional occupation numbers is used. As
in SZ, this mean configuration is usually obtained by distributing the active
electrons in the initial and final states. Therefore the potential obtained is
not optimized to a single configuration. Rather, it is a compromise to
accommodate different configurations. To reduce the error on level
energies due to the use of this less optimized potential, we apply the 
following correction procedure. Before the potential for the mean
configuration is calculated, we obtain an optimized potential for each
configuration, and calculate the average energy of the configuration using
this potential. The average energy of each configuration is recalculated using
the potential optimized for the mean configuration. The difference of the two
average energies are then applied as a correction to the states within that
configuration after the Hamiltonian is diagonalized.

\subsection{Solution of Dirac Equations}
Since the potential depends on the radial orbitals sought, a self-consistent
iteration is required to solve Eq.~(\ref{eq_dirac}). In each iteration,
the orbitals from the previous step are used to derive the
potential. Therefore, one only needs to solve the eigenvalue problem with a
known potential. As is 
standard, we convert Eq.~(\ref{eq_dirac}) into a Schr\"{o}dinger-like
equation by 
eliminating the small component and performing the transformation
\cite{chernysheva99} 
\begin{eqnarray}
\label{eq_transform}
P_a &=& \xi_a(r) F_a(r) \nonumber \\*
\xi_a(r) &=& \sqrt{1+\frac{\alpha^2}{2}\left[\varepsilon_a-V(r)\right]} 
\nonumber \\*
Q_a &=& \frac{\alpha}{2\xi_a^2}\left(\frac{d}{d r}P_a + 
\frac{\kappa}{r}P_a\right),
\end{eqnarray}
Under this transformation, we have
\begin{equation}
\label{eq_schrodinger}
\frac{d^2}{d r^2}F_a(r) + \left\{2\left[\varepsilon_a-U(r)\right] - 
\frac{\kappa(\kappa + 1)}{r^2}\right\}F_a(r) = 0,
\end{equation}
where $U(r)$ is an effective potential defined as
\begin{eqnarray} 
\label{eq_UW}
U(r) &=& V(r) - \frac{\alpha^2}{2}\left\{\left[V(r) - \varepsilon_a\right]^2 
-W(r)\right\} \nonumber \\*
W(r) &=& \frac{1}{4\xi^2(r)}\left[\frac{d^2}{d r^2}V(r) +
\frac{3\alpha^2}{4\xi^2(r)}\left(\frac{d}{d r}V(r)\right)^2 - 
\frac{2\kappa}{r}\frac{d}{d r}V(r)\right].
\end{eqnarray}

We use the standard Numerov method to solve Eq.~(\ref{eq_schrodinger}). 
However, it is customary to perform another transformation before seeking the
solution
\begin{eqnarray}
t &=& t(r) \nonumber\\*
F_a(r) &=& \left(\frac{d t}{d r}\right)^{-1/2} G_a(t),
\end{eqnarray}
where $t(r)$ as a function of radial distance is suitably chosen
so that a uniform grid can be used in the new variable $t$, and the
corresponding transformation on the wavefunction is to bring the differential
equation for $G_a(t)$ to a Schr\"{o}dinger-like form, i.e., without the first
derivative term
\begin{eqnarray}
\label{eq_Ga}
\frac{d^2}{d t^2}G_a(t) &=& \left(\frac{d t}{d r}\right)^{-2}G_a(t)
\Bigg\{\frac{\kappa(\kappa +1)}{r^2} - 2\left[\varepsilon_a-U(r)\right]
\nonumber\\*
&+&\frac{1}{2}\left(\frac{d t}{d r}\right)^{-1}\frac{d^3 t}{d r^3}
-\frac{3}{4}\left(\frac{d t}{d r}\right)^{-2}\left(\frac{d^2 t}{d
r^2}\right)^2\Bigg\}.
\end{eqnarray}

Two types of $t(r)$ have been used in the past. One is a logarithmic
transformation, $t(r) \propto \ln(r)$, which has been used in the MCHF code of
\citet{fischer00}; the other is a hybrid form, $t(r) = c_1 r + c_2 \ln(r)$,
e.g., as used in ATOM \cite{amusia97}. The logarithmic form is not
suitable for 
highly excited and continuum orbitals, because the radial grid interval may
exceed the oscillation period of the wavefunction at large $r$. In the hybrid
form, the grid interval approaches a constant at large $r$. For suitably
chosen coefficients $c_1$ and $c_2$, it can be used in the calculation of
highly excited orbitals and continua with energy below some limit. However,
for free orbitals with sufficiently high energy, solving Eq.~(\ref{eq_Ga})
in a conventional way becomes impractical. We shall use a different approach
for continuum states, namely, the phase-amplitude method as used in
HULLAC. For highly excited bound states, it is easily shown that the
oscillation period of the 
wavefunction is $\propto \sqrt{r}$ at large $r$. We therefore use a modified
hybrid form, $t(r) = c_1\sqrt{r} + c_2\ln(r)$, so that 
one oscillation period contains approximately the same number of grid points
at large distances. The advantage of the modified form
over the linear hybrid form is that for a given number of grid points, the
modified form can cover a larger radial distance than the linear form, which is
important for the calculation of highly excited states. 

The minimum and maximum radial distances, $r_{min}$ and $r_{max}$, in setting
up the radial grid are chosen as 
\begin{eqnarray}
r_{min} &=& 10^{-5}/Z_{eff} \nonumber \\*
r_{max} &=& 500/Z_{eff},
\end{eqnarray}
where $Z_{eff}$ is the residual charge of the atomic ion that the electrons
experience at large $r$. This ensures that $r_{min}$ is well within the
nuclear charge distribution for any atomic system. The value of
$r_{max}$ ensures that for excited states below $n \sim 20$, the bound energies
are less than the Coulomb potential at $r_{max}$. These states have no nodes at
$r > r_{max}$. For states with higher $n$, however, wavefunctions beyond
$r_{max}$ may have additional nodes. Therefore, counting the nodes is no
longer a valid method to pick out the right solution. Moreover, the
wavefunctions can no longer be normalized by calculating their norm with simple
numerical integration, since the contributions beyond $r_{max}$ cannot be
neglected. Actually, such contributions must be included for $n$ well within
20. One may increase $r_{max}$ when highly excited states are needed. One
must also increase the number of grid points as well to
ensure the accuracy of numerical integration. However, wavefunctions
at large radial distances are rarely needed, either because the interaction
operators are negligible, or because the states that interact with
such highly excited orbitals have negligible amplitudes at large distances. 
In the present program, the low-$n$ and high-$n$ states are treated
differently. The dividing $n_0$ is determined by the choice of $r_{max}$,
specifically, $n_0
= 0.5*\sqrt{Z_{eff}r_{max}}$. For $n \le n_0$, the orbitals are found by
outward and inward integration of Eq.~(\ref{eq_Ga}) with zero amplitudes at
both ends, and matching at the outer classical turning point. Node counting is
used to pick out the appropriate solution corresponding to the quantum numbers
$n$ and $l$. The wavefunctions are then normalized by 
numerical integration. For $n > n_0$, Eq.~(\ref{eq_Ga}) is integrated
outward until $r=r_{core}$, where the potential has reached its asymptotic
Coulomb value. For $r > r_{core}$, the wavefunction is the exponentially
decaying Whittaker function 
\begin{equation}
y_5(\nu, \lambda, \rho) = W_{\nu, \lambda+1/2}(2\rho/\nu),
\end{equation}
where $\nu^2 = -\frac{1}{2}Z_{eff}^2/\varepsilon$, $\rho = Z_{eff}r$, and
$\lambda = l$ in the non-relativistic limit \cite{seaton58}. In the
relativistic case, the asymptotic behavior of the effective potential is
modified according to Eq.~(\ref{eq_UW}), and corresponds to 
\begin{eqnarray}
Z_{eff}^\prime &=& Z_{eff}(1+\alpha^2\varepsilon) \nonumber\\*
\nu^2 &=& -\frac{Z_{eff}^{\prime 2}}{2\varepsilon
\left(1+\frac{1}{2}\alpha^2\varepsilon\right)} \nonumber\\*
\lambda(\lambda+1) &=& l(l+1) - (Z_{eff}\alpha)^2 .
\end{eqnarray}
The Whittaker function and its derivative at $r_{core}$ are calculated
using the program of \citet{thompson85} and matched with the solution
obtained by outward integration. The program of \citet{thompson85} uses
quadrupole precision floating point numbers in the calculation of 
hypergeometric function to achieve the desired accuracy. Since not all computer
platforms implement the quadrupole precision floating point numbers natively,
we have modified the program to use a pure software quadrupole precision
arithmetics to increase the portability of the
program. Since for highly excited states, the quantum defect is 
small, the solution with $\nu \sim n$ is picked out without node counting. To
normalize the wavefunction, we note that the correct normalization is given by
\cite{seaton58}
\begin{equation}
\label{eq_norm}
F(r > r_{core}) = K(\nu_n, \lambda)y_5(\nu_n, \lambda, \rho),
\end{equation}
where 
\begin{equation}
K(\nu,\lambda) =
Z_{eff}^{\prime 1/2}\left[\zeta(\nu_n)\nu_n^2\Gamma(\nu_n+\lambda+1)
\Gamma(\nu_n-\lambda)\right]^{-1/2}, 
\end{equation}
and 
\begin{equation}
\zeta(\nu_n) = 1 + \frac{d\mu}{d\nu},
\end{equation}
where $\mu$ is the quantum defect. For high $n$ states we are concerned with,
$\zeta(\nu_n) = 1$ is a very good approximation.

\subsection{Angular Integration and Hamiltonian Matrix Elements}
In Eq.~(\ref{eq_hamilton}), the first term of the Hamiltonian is a
one-electron operator, while the second term is a two-electron operator. The
traditional method of evaluating their matrix elements is to expand them into a
sum, with each term being a product of an angular part and a radial part. The
angular part is then calculated using Racah algebra. In doing so, the initial
and final basis states need to be recoupled, which are often carried out by
the recoupling program of \citet{grant76}. Recently, \citet{gaigalas97}
proposed a new method of performing angular integration, which is based on the
second quantization form of the operators and extends the use of Racah algebra
to the quasi-spin space. In this method, instead of recoupling basis states,
one recouples the creation and annihilation operators with the help of Racah
algebra. The main advantage of this method 
is that there are only two creation and two annihilation operators in the
two-electron interaction, while for the one-electron interaction, there is
only one creation and one annihilation operator. Therefore, at most four
angular momenta 
are involved in the recoupling independent of the shell structure of the
basis states. In the conventional method, the recoupling of basis states can
be quite complicated for complex configurations. The present code adopts the
new method and the program of \citet{gaigalas01} is used for the reduced
matrix elements of creation and annihilation operators.

\subsubsection{One-electron Operators}
The one-electron operator in the Hamiltonian is a scalar, however, we treat a
general tensorial operator $O^L_M = \sum_i o^L_M(i)$ in this section since the
calculation of 
radiative transition rates involves tensors. In second quantization form,
$O^L_M$ may be expressed as 
\begin{equation}
O^L_M = \sum_{\hat{\alpha}\hat{\beta}} a^{\dagger}_{\hat{\alpha}} 
a_{\hat{\beta}} <\hat{\alpha}|o^L_M|\hat{\beta}>,
\end{equation}
where $\hat{\alpha}$ and $\hat{\beta}$ denotes a single electron state
$n\kappa m$.  
$a^{\dagger}$ is the creation operator and $a$ is the annihilation
operator. Using Wigner-Ekart theorem for the matrix elements of $o^L_M$ we
have
\begin{equation}
O^L_M = \sum_{\alpha\beta} Z^L_M(\alpha,\beta) 
<\alpha||o^L||\beta>,
\end{equation}
where $<\alpha||o^L||\beta>$ denotes the reduced matrix element, and
$\alpha$ and $\beta$ denote only quantum numbers $n\kappa$. The
summation over $m$ is already contained in $Z^L_M$, which is defined as
\begin{equation}
Z^L_M(\alpha,\beta) = 
-[L]^{-1/2}\left[a^{\dagger}_{\hat{\alpha}} \times
\tilde{a}_{\hat{\beta}}\right]^L_M,
\end{equation}
where $[L] = 2L+1$, and $\tilde{a}_{\hat{\beta}} = (-1)^{j_\beta -
m_\beta}a_{-\hat{\beta}}$ 
with $-\hat{\beta}$ being understood as the single electron state
$n_\beta\kappa_\beta -\!m_\beta$, i.e., having the magnetic quantum number
negated. Such a transformation is necessary   
since it is $\tilde{a}_{\hat{\beta}}$ that form an irreducible tensorial set
with rank 
$j_\beta$ \cite{judd67}. The tensorial coupling has the usual meaning. The
angular integration is equivalent to the 
evaluation of the reduced matrix elements of $Z^L_M$ between basis states
\cite{gaigalas97}.

\subsubsection{Two-electron Operators}
After some algebraic manipulation \cite{barshalom:1988a}, the
electro-static interaction between electrons can be written as 
\begin{equation}
\label{eq_2e}
\sum_{i<j} \frac{1}{r_{ij}} = \frac{1}{2}
\sum_{\alpha\beta\gamma\delta}
\sum_{k}\Bigg\{Z^k(\alpha,\gamma)\cdot Z^k(\beta,\delta)
-(-1)^{j_\alpha-j_\beta}[j_\alpha]^{-1/2}Z^0_0(\alpha\delta)\Bigg\}
X^k(\alpha\beta;\gamma\delta),
\end{equation}
where $Z^k(\alpha,\gamma)\cdot Z^k(\beta,\delta)$ denotes the scalar product
of the two tensors, and 
\begin{equation}
X^k(\alpha\beta;\gamma\delta) = <\alpha||C^k||\gamma><\beta||C^k||\delta>
R^k(\alpha\beta;\gamma\delta),
\end{equation}
where $C^k$ is the normalized spherical harmonic tensor as defined in
\citet{cowan81}, and $R^k$ is the generalized Slater integral
\begin{equation}
R^k(\alpha\beta;\gamma\delta) = \int \frac{r_<^k}{r_>^{k+1}}
\rho_{\alpha\gamma}(r_1)\rho_{\beta\delta}(r_2)d r_1 d r_2.
\end{equation}
The calculation of the 
matrix elements of $Z^k(\alpha,\gamma)\cdot Z^k(\beta,\delta)$ follows
\citet{gaigalas97}.

\subsection{Large Scale Configuration Interaction}
It is sometimes necessary to carry out calculations with a very large number of
interacting configurations. Since the dimension of the Hamiltonian matrix
grows very rapidly as the number of the interacting configurations
increases. Such large
scale calculations easily become unrealistic due to the insufficient CPU time
and memory. In the present code, we use an approximate procedure to deal with
such situations. In most cases, one is only interested in a small subset of the
energy levels resulting from a large configuration space. The
majority of the configurations are included only to provide mixing to the
desired levels. Therefore, we partition the configuration space into two
groups. The first is the main group whose diagonalization provides a good
zeroth order approximation to the desired energy levels. The second is a
perturbing group whose interaction with the main group is weak, though not
negligible. The Hamiltonian matrix can therefore be partitioned
correspondingly
\begin{equation}
H = \left(\begin{array}{cc}H_1 & B^{\dagger}\\B & H_2\end{array}\right),
\end{equation}
where $H_1$ is the matrix resulting from the main group, and $H_2$ from the
perturbing group. The matrix $B$ and its Hermitian conjugate $B^{\dagger}$
result from the interaction between the two groups. In our approximating
method, only 
the matrix $H_1$, $B$, $B^{\dagger}$, and the diagonal elements of $H_2$ are
retained, i.e., the interaction within the perturbing group is ignored. This
can save large amount of computing time and storage space when the dimension of
$H_2$ is much larger than that of $H_1$. To avoid diagonalizing the large
matrix $H$ directly, we partition the eigenvectors of $H$ accordingly
\begin{equation}
\left(\begin{array}{cc}H_1 & B^{\dagger} \\ B & H_2\end{array}\right) 
\left(\begin{array}{cc}Q_{11} & Q_{12} \\ Q_{21} & Q_{22} \end{array}\right) 
= \left(\begin{array}{cc}Q_{11} & Q_{12} \\ Q_{21} & Q_{22} \end{array}\right)
\left(\begin{array}{cc} D_1 & 0 \\ 0 & D_2 \end{array}\right),
\end{equation}
where the columns of matrix $Q$ are the eigenvectors of $H$, $D_1$ and $D_2$
are diagonal matrices whose elements are the eigenvalues. We are only
interested in the eigenvalues and eigenvectors corresponding to the subspace
of the main group, i.e, the matrices $D_1$, $Q_{11}$, and $Q_{21}$. These
matrices satisfy the following equations
\begin{eqnarray}
\label{eq_partition}
H_1 Q_{11} + B^{\dagger}Q_{21} &=&  Q_{11} D_1 \nonumber \\*
B Q_{11} + H_2 Q_{21} &=& Q_{21} D_1 .
\end{eqnarray}
The second relation of Eq.~(\ref{eq_partition}) results in 
\begin{equation}
\label{eq_Q21}
Q_{21}^{i} = (D_1^{i} - H_2)^{-1}B Q_{11}^{i},
\end{equation}
where $Q_{21}^{i}$ and $Q_{11}^{i}$ are the $i$-th columns of $Q_{21}$ and
$Q_{11}$, $D_1^{i}$ is the $i$-th diagonal element of $D_1$. Combined with the
first relation of Eq.~(\ref{eq_partition}), the eigenvalues and
eigenvectors can 
be solved iteratively starting with $Q_{21} = 0$, $D_1$ and $Q_{11}$ being the
eigenvalues and eigenvectors of $H_1$. During each step, one needs to
diagonalize the matrix $Q_{11}^{T} H_1 Q_{11} + Q_{11}^{T}B^{\dagger}Q_{21}$,
which has the same dimension as $H_1$. Since $H_2$ is diagonal, the matrix
inversion in Eq.~(\ref{eq_Q21}) is trivial. Practically, if the partition
between the two groups is carefully chosen, a single iteration yields
sufficiently accurate eigenvalues and eigenvectors.

\section{Radiative Transition Rates}
The radiative transition rates are calculated in the single multipole
approximation. This means that the interference between different multipoles
is not taken into account, although rates corresponding to arbitrary
multipoles can be calculated. For a given multipole operator $O^L_M$, and
initial and final states of 
the transition $\psi_i = \sum_\nu b_{i\nu}\Phi_\nu$ and $\psi_f = \sum_\mu
b_{f\mu}\Phi_\mu$, the line strength of the transition is 
\begin{eqnarray}
S_{fi} &=& \left|<\psi_f||O^L_M||\psi_i>\right|^2 \nonumber\\*
&=& \left|\sum_{\mu\nu}b_{f\mu}b_{i\nu}\sum_{\alpha\beta}
<\Phi_\mu||Z^L_M(\alpha,\beta)||\Phi_\nu><\alpha||C^L||\beta>
M^L_{\alpha\beta}\right|^2 ,
\end{eqnarray} 
where $M^L_{\alpha\beta}$ is the radial part of the single-electron multipole
operator as defined by \citet{grant74}.
The weighted oscillator strength and transition rates are given by 
\begin{eqnarray}
gf_{fi} &=& [L]^{-1}\omega(\alpha\omega)^{2L-2} S_{fi} \\*
gA_{fi} &=& 2\alpha^3 \omega^2 gf_{fi},
\end{eqnarray}
where $\omega = E_i - E_f$ is the transition energy. 

$M^L_{\alpha\beta}$ may be calculated using the fully relativistic
expressions of \citet{grant74}. However, in most cases (except for M1
transitions, where the use of the fully relativistic expressions is
essential), their non-relativistic 
limits are sufficiently accurate, which has the advantage that these operators
depend on the transition energy in a trivial manner. 


\section{Electron Impact Excitation}
\label{sec:eie}
\subsection{Introduction}
Two classes of methods are commonly used in the calculation of electron impact
excitation (EIE) cross
sections. The first is based on a set of close-coupling (CC) equations, which
takes into account the coupling of various excitation channels
\cite{seaton75}. In these methods, resonances can be included in a
natural way by including the coupling to closed channels. Several
implementations of this method exist. The most widely used is the R-matrix
code developed by a group at the Queens University of Belfast
\cite{berrington95}. The second class of methods is based on the first order
Born approximation, which assumes independent excitation channels. The
coupling to closed channels, which results in resonances, may be included in a
perturbative fashion \cite{eissner72}. Different variants exist according to
the different treatments of the continuum wavefunctions. The plane-wave (PW)
Born approximation uses an unperturbed plane wave for the free orbital. The
Coulomb-wave (CW) Born approximation takes into account the distortion of the
continuum due to a pure Coulomb potential. The most accurate of this class is
the DW Born approximation, in which the free orbitals are
calculated in a more realistic potential taking into account the electronic
structure of the target ion.
The majority of the computer programs in this class implement the DW
approximation, since it yields significantly better results than the PW and CW
methods with minimal increase in the complexity. Many DW codes are in use
today. For example, the non-relativistic DW code from University College
London \cite{eissner98}, the relativistic code of \citet{hagelstein87}, the
HULLAC package \cite{barshalom:1988a}, the code by \citet{zhang89}, and that of
\citet{chen96}, just to name some. The present DW implementation in FAC is
in principle similar to any of the relativistic codes listed above.

In the present implementation, the factorized formula of the collision
strength is used, which separates the angular integration from the radial
integrals, and permits efficient interpolation of the radial integrals. To
avoid the difficulties in the radial integration arising from the oscillatory
behavior of the continuum wavefunctions, we use a different technique from
that for bound orbitals to solve the Dirac equation for free electrons. In the
inner region where the wavefunction is not oscillatory or when one oscillation
period contains 
a sufficient number of radial grid points to ensure the accuracy of numerical
integration, we use the standard Numerov method to integrate the equation 
outward. In the outer region, a phase-amplitude method is used. The position
where the outer solution is matched to the inner solution depends on the
energy and angular momentum of the continuum orbital. This method allows us to
evaluate the radial integrals involving the continuum orbitals to sufficiently
large radial distance without using a prohibitively dense grid. The collision
strength is calculated using the standard partial wave summation. A
quasi-logarithmic grid is constructed for the orbital angular momenta of the
free electrons. Detailed calculations are only carried out on this
grid. Partial collision strengths for all other orbital angular momenta are
obtained by interpolation. Very large
angular momentum contributions to the allowed transitions
are taken into account by the rapid Coulomb-Bethe approximation
\cite{burgess74}. 

\subsection{Factorized Collision Strength}
The EIE cross subsection $\sigma_{01}$ from the initial state $\psi_0$ to the
final state $\psi_1$ can be expressed in terms of the collision strength
$\Omega_{01}$ as (in atomic units, which we shall use throughout the paper) 
\begin{equation}
\sigma_{01} = \frac{\pi}{k_0^2g_0}\Omega_{01},
\end{equation}
where $g_0$ is the statistical weight of the initial state, and $k_0$ is the
kinetic momentum of the incident electron, which is related to the energy
$\varepsilon_0$ by
\begin{equation}
k_0^2 = 2\varepsilon_0\left(1+\frac{\alpha^2}{2}\varepsilon_0\right),
\end{equation}
where $\alpha$ is the fine structure constant. The continuum wavefunction
is normalized so that the large component has an asymptotic amplitude of
$\sqrt{k/\varepsilon}$, which reduces to $\sqrt{2/k}$ in the non-relativistic
limit, or equivalently,
\begin{equation}
\int_0^\infty \left[P_{\varepsilon(r)}P_{\varepsilon^\prime}(r)
+Q_{\varepsilon(r)}Q_{\varepsilon^\prime}(r)\right]d r = 
\pi\delta(\varepsilon - \varepsilon^\prime),
\end{equation}
where $\varepsilon$ and $k$ are the energy and kinetic momentum of the
orbital, $P_\varepsilon$ and $Q_\varepsilon$ are the large and small
components of the continuum wavefunction. The collision strength can be
written as
\begin{equation}
\Omega_{01} = 2\sum_{\kappa_0\kappa_1}\sum_{J_T}[J_T]
|<\psi_0\kappa_0,J_TM_T|\sum_{i<j}\frac{1}{r_{ij}}|\psi_1\kappa_1,J_TM_T>|^2,
\end{equation}
where $\kappa_0$ and $\kappa_1$ are the relativistic angular quantum numbers of
the incident and scattered electrons, $J_T$ is the total angular momentum
when the target state is coupled to the continuum orbital, $M_T$ is the
projection of the total angular momentum, and  $[J] = 2J+1$. 
Following \citet{barshalom:1988a}, this expression can be simplified to give
\begin{equation}
\label{eq_cs}
\Omega_{01} = 2\sum_{k}\sum_{\alpha_0\alpha_1\atop\beta_0\beta_1}
Q^k(\alpha_0\alpha_1;\beta_0\beta_1)
<\psi_0||Z^k(\alpha_0,\alpha_1)||\psi_1>
<\psi_0||Z^k(\beta_0,\beta_1)||\psi_1>,
\end{equation}
where 
\begin{equation}
\label{eq_Qk}
Q^k(\alpha_0\alpha_1;\beta_0\beta_1) = \sum_{\kappa_0\kappa_1}[k]^{-1}
P^k(\kappa_0\kappa_1;\alpha_0\alpha_1)P^k(\kappa_0\kappa_1;\beta_0\beta_1),
\end{equation}
and 
\begin{equation}
\label{eq_Pk}
P^k(\kappa_0\kappa_1;\alpha_0\alpha_1)=
X^k(\alpha_0\kappa_0;\alpha_1\kappa_1)+\sum_{t}
(-1)^{k+t}[k]\sixj{j_{\alpha_0}}{j_1}{t}{j_0}{j_{\alpha_1}}{k}
X^t(\alpha_0\kappa_0;\kappa_1\alpha_1),
\end{equation}
where $X^k$ and the operator $Z^k(\alpha,\beta)$ are defined in
Sec.~\ref{sec:structure}. 
 
The importance of Eq.~(\ref{eq_cs}) is that the angular and radial integrals
are completely factorized. The radial integrals
$Q^k(\alpha_0\alpha_1;\beta_0\beta_1)$ with the same set of bound orbitals
$\alpha_0$, $\alpha_1$, $\beta_0$, and $\beta_1$ may appear in many
transitions. These integrals also depend on the energy of incident
and scattered electrons, or for a fixed scattered electron energy, they depend
on the excitation energy of the transition $\Delta E$. However, as noted by
\citet{barshalom:1988a}, the dependence on $\Delta E$  is rather weak. For
dipole forbidden radial integrals, $Q^k \propto \Delta E$, and for dipole
allowed integrals, $Q^k \propto \ln\Delta E$ approximately holds over a wide
range of transition energies. Therefore, for a given scattered electron energy,
$Q^k$ may be calculated at a few values of $\Delta E$, and the integral at the
actual transition energy can be interpolated from these few values. In
practice, 
usually a three-point grid spanning the entire transition energy range for
a given array of excitations yields sufficiently accurate results. The
dependence of $Q^k$ on $\varepsilon_1$, the scattered electron energy,
although not as simple as that on $\Delta E$, is still rather smooth, and has
known asymptotic behavior at large energies according to the type of the
transition. We use interpolation on $\varepsilon_1$ as well with a few more
points. The calculation of $<\psi_0||Z^k||\psi_1>$ is the same as that
involved in the radiative transition rates, see Sec.~\ref{sec:structure}.

\subsection{Solution of the Dirac Equation for The Continuum}
In FAC, the continuum orbitals are obtained by solving the Dirac equations with
the same central potential as that for bound orbitals. However, in obtaining the
potential, one may optionally add a high lying subshell to the mean
configuration to account for the fact that the continuum  wavefunctions
experience the screening of one additional electron at large distances. Such a
high  lying subshell has little effect on the bound orbitals because the
deviation of the potential from its correct asymptotic value starts at very
large distance where the bound orbitals have exponentially decayed.  After
transforming the Dirac equations to a second order Schr\"{o}dinger-like
equation, and adopting the same radial grid as in the solution of bound orbitals
(see Sec.~\ref{sec:structure}), the transformed large component, $F(r)$,  of the
free orbital satisfies 
\begin{eqnarray}
\label{eq_schrodinger2}
\frac{d^2}{d r^2}F(r) + \left\{2\left[\varepsilon-U(r)\right] - 
\frac{\kappa(\kappa + 1)}{r^2}\right\}F(r) = 0,
\end{eqnarray}
where $\varepsilon$ is the energy of the free electron, and the boundary
condition at infinity is replaced by the requirement that the
original large component $P(r)$  has the asymptotic amplitude of
$\sqrt{k/\varepsilon}$.  

In solving Eq.~(\ref{eq_schrodinger2}), the radial grid is divided into two
regions. In the inner region, where the wavefunction is not oscillatory, or
the oscillation period is large enough to contain a sufficient number of grid
intervals (e.g., more than 16), we use the standard Numerov method to
integrate the equation outward. Beyond some point $r = r_c$, which depends on
the energy and angular momentum of the continuum sought, the oscillation
period of the wavefunction becomes too small for the direct integration to be
accurate. At that point, we switch to a phase-amplitude method, in which
$F(r)$ is written as
\begin{equation}
F(r) = A\frac{1}{\eta^{1/2}(r)} \sin\phi(r),
\end{equation}
where the constant $A$ is chosen to ensure the appropriate normalization.
$\phi(r)$ and $\eta(r)$ satisfy  
\begin{eqnarray}
\label{eq_pham}
\phi(r) &=& \int_0^{r} \eta(s) d s \nonumber \\*
\eta^2(r) &=& \eta^{1/2}\frac{d^2}{d r^2}\eta^{-1/2} + \omega^2(r) ,
\end{eqnarray}
where
\begin{equation}
\omega^2(r) = 2\left[\varepsilon-U(r)\right] - 
\frac{\kappa(\kappa + 1)}{r^2} .
\end{equation}
For $r > r_c$, Eq.~(\ref{eq_pham}) can be easily solved by iteration
starting from the first order WKB approximation $\eta(r) = \omega(r)$. In
fact, in most cases, the first order approximation itself is sufficiently
accurate. The inner and outer solutions are matched at $r_c$ by requiring the
continuity of $F(r)$ and its first derivative. 

\subsection{Evaluation of Radial Integrals}
The evaluation of Slater integrals reduces to the calculation of the following
type of integrals 
\begin{equation}
I = \int_0^\infty P_a(r)f(r)P_b(r)d r,
\end{equation}
where $P_a(r)$ and $P_b(r)$ may be large or small components of either bound
or continuum orbitals, and $f(r)$ is a smooth function of $r$. If both
wavefunctions are from bound orbitals, direct numerical integration is
used. 

If one of them is from a continuum orbital, we divide the integration
range into two regions. In the first region, $r < r_c$, the integration
proceeds as in the bound-bound case. In the $r > r_c$ region, the integral is
of the type 
\begin{equation}
I_1 = \int g(r)\sin\phi(r) d r ,
\end{equation}
or 
\begin{equation}
I_2 = \int g(r)\cos\phi(r) d r ,
\end{equation}
where $g(r)$ is a smooth function of $r$. Since the phase $\phi$ is also a
smooth function of $r$, we evaluate $I_1$ as
\begin{equation}
I_1 = \int \tilde{g}(\phi)\sin\phi d\phi,
\end{equation}
where 
\begin{equation}
\tilde{g}(\phi) = g(r)\frac{d r}{d\phi},
\end{equation}
which is a smooth function of $\phi$. Using its values at each grid point, it
is represented by a cubic spline interpolation function. Therefore within each
grid interval, it is a third order polynomial of $\phi$. $I_1$ is evaluated by
integrating $\int \phi^n\sin\phi d\phi$ analytically, where $n$ = 0, 1, 2, or
3.  The evaluation of $I_2$ is similar, replacing $\sin\phi$ with $\cos\phi$. 

If both wavefunctions are from the continuum, the integration range is divided
into three regions. In the first region, $r < \mbox{min}(r_{c1}, r_{c2})$, the
integration proceeds as in the bound-bound case. In the second region,
$\mbox{min}(r_{c1}, r_{c2}) < r < \mbox{max}(r_{c1}, r_{c2})$, the integration
proceeds as in the bound-free 
case. In the last region, $r > \mbox{max}(r_{c1}, r_{c2})$, both wavefunctions
are in the phase-amplitude form, the integrals are of the type 
\begin{equation}
I = \int g(r) \sin\phi_1(r)\sin\phi_2(r) d r,
\end{equation}
or similar ones where one or both sine functions are replaced by cosine
functions. Such integrals are transformed to the sum of two terms
\begin{eqnarray}
I^+ &=& \int g^+(r)\cos\phi^+(r)d r \nonumber \\*
I^- &=& \int g^-(r)\cos\phi^-(r)d r ,
\end{eqnarray}
or variants where the cosine function is replaced by the sine function. In the
above equation, $\phi^+ = \phi_1 + \phi_2$ and $\phi^- = \phi_1 -\phi_2$,
respectively. These 
integrals are evaluated similar to the bound-free integrals except when the
energies of two continuum orbitals are very close so that $\phi^-$ is very
small, in 
which case, the integrals containing $\phi^-$ are calculated directly in the
radial variable $r$. 

\subsection{Quasi-relativistic Approximation}
In most cases, it is not necessary to solve the continuum radial equation fully
relativistically. As noted by \citet{zhang89}, the quasi-relativistic
approximation, in which the two wavefunctions with the same $l$ but different
$j$ are treated in the same way, and the small components are neglected, often
yields final collision strengths very close to the fully relativistic results
for the nuclear charge as large as 74 and collision energy as high as 30 keV.
Even for $Z = 92$, the differences are mostly within 15--20\%. 

In the present code, we incorporate an option for such approximations as
well. In fact, the 
approximation can be invoked only for $l > l_{qr}$, where $l_{qr}$ is some
suitable value, since one would expect the approximation to get better for
higher $l$. Because
the quasi-relativistic approximation is very good for all low to mid $Z$
elements and saves 
considerable amount of computing time, $l_{qr}$ is chosen to be 0 by default,
i.e., all 
partial waves are treated in quasi-relativistic approximation. For high $Z$
elements, one needs to increase $l_{qr}$ or turn off the quasi-relativistic
approximation completely.

\subsection{Partial Wave Summation}
The radial integral $Q^k$, Eq.~(\ref{eq_Qk}), is expressed as a sum of
partial waves. The summation is evaluated as following
\begin{eqnarray}
Q^k &=& \sum_{l_0}\sum_{l_1} \tilde{Q}^k(l_0,l_1) \nonumber\\*
\tilde{Q}^k(l_0,l_1) &=& \sum_{j_0,j_1}
[k]^{-1}P^k(\kappa_0\kappa_1;\alpha_0\alpha_1)
P^k(\kappa_0\kappa_1;\beta_0\beta_1),
\end{eqnarray}
where the summation over $\kappa$ is explicitly written out as the summation
over the orbital and total angular
momenta. $\tilde{Q}^k(l_0,l_1)$ is not calculated for every possible $l_0$
and $l_1$. A suitable grid is chosen for $l_0$ from 0 to $l_{max}$. For each 
$l_0$ in this grid, $\tilde{Q}^k(l_0) = \sum_{l_1}\tilde{Q}^k(l_0,l_1)$ is
calculated, where the values of $l_1$ are limited by the triangular
relations. $\tilde{Q}^k(l_0)$ is a smooth function of $l_0$ especially when 
$l_0$ is relatively large. Therefore, for any $l_0$ that is missing in the
grid, its value can be obtained by interpolation. The $l_0$ grid used
in the code may be specified in the input. However, the default grid is
usually sufficient, which contains every value from $l_0 = 0$ to 5, after
that, the interval between the successive points is doubled every two points,
i.e., it is almost logarithmic. 

The value of $l_{max}$ needed for the desired accuracy varies widely depending
on the type of the transition and the collision energy. Many programs choose
an arbitrary value for $l_{max}$, and the convergence is not guaranteed for
all transitions at all energies. In the present program, 
$l_{max}$ is not fixed \textit{a priori}. At each point in the $l_0$ grid, an
estimate of the contributions from higher partial waves is made according to
the method described below. The relative uncertainty in the resulting
collision strength is also estimated if the truncated sum plus the estimated
higher partial wave contribution is taken to be the final result. Once this
relative uncertainty becomes less than a prescribed accuracy (usually 5\%),
the summation is truncated at this point.

For strictly forbidden transitions, which have non-zero collision strengths
only due to the exchange interaction, the high partial wave contributions are
not important, no special difficulties arise regarding the convergence of the
summation. 
For allowed transitions (including all multipole types), we use the
Coulomb-Bethe approximation to estimate the high partial wave contributions
and calculate the uncertainties in such estimates. In this
method, the exchange term in $P^k$ is neglected, and the direct term is
approximated by 
\begin{equation}
\label{eq_PkApprox}
P^k_{CB}(\kappa_0\kappa_1;\alpha_0\alpha_1) = 
M_k(\alpha_0\alpha_1)R_k(\kappa_0\kappa_1),
\end{equation}
where 
\begin{eqnarray}
M_k(\alpha_0\alpha_1) &=& <\alpha_0||C^k||\alpha_1>
\int (P_{\alpha_0}P_{\alpha_1}+Q_{\alpha_0}Q_{\alpha_1})r^kd r \nonumber\\*
R_k(\kappa_0\kappa_1) &=&
\int (P_{\kappa_0}P_{\kappa_1}+Q_{\kappa_0}Q_{\kappa_1})
\frac{1}{r^{k+1}}d r.  
\end{eqnarray}
Note that $M_k$ only depends on the target orbitals, and $R_k$ only depends on
the continua. The corresponding $\tilde{Q}^k(l_0,l_1)$ under this
approximation is 
\begin{equation}
\tilde{Q}^k_{CB}(l_0, l_1) = M_k(\alpha_0\alpha_1)M_k(\beta_0\beta_1)
\sum_{j_0j_1}[k]^{-1}R_k^2(\kappa_0\kappa_1),
\end{equation}
and 
\begin{equation}
\tilde{Q}^k_{CB}(l_0)=\sum_{l_1}\tilde{Q}^k_{CB}(l_0,l_1). 
\end{equation}
The ratio
\begin{equation}
\gamma(l_0) = \frac{\tilde{Q}^k_{CB}(l_0+1)}{\tilde{Q}^k_{CB}(l_0)}
\end{equation}
only depends on the continuum orbitals and approaches a constant,
$\gamma_{\infty}=\varepsilon_1/\varepsilon_0$, at large $l_0$. 

For $k \ge 2$,
the convergence to this asymptotic value is relatively fast. Therefore, we may
approximate the summation over $l_0$ by
\begin{equation}
\sum_{l_0>l_{max}}\tilde{Q}^k_{CB}(l_0) =
\tilde{Q}^k_{CB}(l_{max})\frac{\gamma_{\infty}}{1-\gamma_{\infty}}. 
\end{equation}
The difference between $\gamma_{\infty}$ and the value of 
$\tilde{Q}^k(l_0+1)/\tilde{Q}^k(l_0)$ calculated with the actual wavefunctions
may be used to calculate the uncertainty introduced in this
approximation. Suppose this 
difference is $\delta$, the absolute uncertainty is given by
\begin{equation}
\Delta = \tilde{Q}^k_{CB}(l_{max})\frac{\delta}{(1-\gamma_{\infty})^2}.
\end{equation}
This value is compared with the desired accuracy to determine whether the
summation should be truncated.

For $k = 0$ and 1, however, $\gamma_\infty$ is not reached until $l_0 \gg
\varepsilon_1/(\varepsilon_0-\varepsilon_1)$, which can be very large, if the
transition energy is very small or the collision
energy is high. For such cases, we make use of the well known non-relativistic
recursive relations which connect
$\tilde{Q}^k_{CB}(l_0,l_1)$ with different $l_0$ to calculate the ratio
$\gamma(l_0)$ until it converges to 
$\gamma_{\infty}$ \cite{burgess74}. These ratios are then used to estimate
the high partial wave contributions and the uncertainties in these estimates.

\subsection{Electron Collisional Excitation Between Magnetic Sublevels}
\subsubsection{Introduction}
Electron collisional excitation cross sections are required for calculation of
the level population and line emission of hot plasmas, such as those
encountered in astrophysical environments. In many cases, only the total
excitation cross sections are of importance, at least when the electron
distribution functions are isotropic. However, there exist situations where
aligned excitation produces polarized line emission. Suggestions have been
made to use such polarized light to study beam-plasma interactions in solar
flares \citep{haug72, haug81} and the properties of tokamak plasmas
\citep{fujimoto96}. Line polarization is also an important factor to take into
account in the analysis of laboratory spectroscopic data involving a
directional electron beam, such as electron beam ion traps
\citep{beiersdorfer96, takacs96}. 

In order to determine the degree of polarization and angular distribution of
emitted lines driven by electron collisional excitation, one needs detailed
cross sections between the magnetic sublevels of lower and upper states. Several
authors have made calculations of such cross sections involving magnetic
sublevels. \citet{mitroy:1988a} and \citet{mitroy:1988b} used a nonrelativistic
LS-coupling approach. \citet{inal87} made distorted-wave (DW) calculations using
nonrelativistic radial wavefunctions, but they included intermediate-coupling
effects by transforming the reactance matrices from LS-coupling to relativistic
pair-coupling scheme \citep{eissner72, saraph:1978a}. \citet{zhang90} performed
the first fully relativistic DW calculations of magnetic sublevel collision
strengths with a modified version of their previous program for total cross
sections.

Since the pioneering work of \citet{barshalom:1988a}, the factorization theory
of DW collisional excitation is in wide use for calculating total cross
sections. This theory enables one to obtain a complete collisional excitation
array without calculating a large number of radial integrals by using
appropriate interpolation procedures. However, DW calculations of magnetic
sublevel collision strengths have not utilized such factorization-interpolation
techniques. In this paper, we generalize the theory of \citet{barshalom:1988a}
to the excitation of magnetic sublevels and present its computer implementation.

\subsubsection{Factorization Theory}
The scattering amplitude $B_{m_{si}}^{m_{sf}}$ can be written as
(given in Zhang, Sampson, and Clark, 1990, PRA, 41, 198)
\begin{eqnarray}
B_{m_{si}}^{m_{sf}}&=&\frac{2\pi}{k_i}
  \sum_{l_i,m_{li},j_i,m_i \atop l_f,m_{lf},j_f,m_f}
  (i)^{l_i-l_f+1}\exp[i(\delta_i+\delta_f)]
  Y_{l_i}^{m_{li}*}(\hat{\textbf{k}}_i)
  Y_{l_f}^{m_{lf}}(\hat{\textbf{k}}_f) \nonumber\\
&&\times C(l_i\frac{1}{2}m_{li}m_{si};j_im_i)
  C(l_f\frac{1}{2}m_{lf}m_{sf};j_fm_f)
  T(\alpha_i,\alpha_f)
\end{eqnarray}
where $T(\alpha_i,\alpha_f)$ is the transition matrix elements in the 
representation where the free electrons are uncoupled to the targets,
\begin{equation}
\alpha_i=k_i\tilde{j_i}m_iJ_iM_i,\ \alpha_f=k_f\tilde{j_f}m_fJ_fM_f,
\end{equation}
where $\tilde{j}$ denotes $\{l,j\}$.
The differential cross section is 
\begin{equation}
\frac{d\sigma}{d\hat{k_f}}=\left|B_{m_{si}}^{m_{sf}}\right|^2. 
\end{equation}
Choosing the direction of the incident electron as the $z$ axis,
integrating over $\hat{k_f}$, summing over $m_{sf}$, and averaging over
$m_{si}$ gives
\begin{eqnarray}
\label{eq_cross0}
\sigma(J_fM_f,J_iM_i)&=&\frac{\pi}{2k_i^2}
  \sum_{\tilde{j_i},\tilde{j_i^\prime},\tilde{j_f} \atop m_{si},m_f}
  (i)^{l_i-l_i^\prime}\exp[i(\delta_i-\delta_{i^\prime})]
  ([l_i][l_i^\prime])^\frac{1}{2} \nonumber\\
&&\times
  (-1)^{j_i+j_i^\prime+2m_i}([j_i][j_i^\prime])^{\frac{1}{2}}
  \pmatrix{j_i & \frac{1}{2} & l_i \cr -m_i & m_{si} & 0 \cr}
  \pmatrix{j_i^\prime & \frac{1}{2} & l_i^\prime \cr -m_i & m_{si} & 0 \cr}
  \nonumber \\
&&\times
  T(\alpha_i,\alpha_f)T^{*}(\alpha_i^\prime,\alpha_f)
\end{eqnarray}
In the first order perturbation theory
\begin{equation}
T(\alpha_i,\alpha_f)=-2i<J_iM_i\tilde{j_i}m_i|V|J_fM_f\tilde{j_f}m_f>
\end{equation}
where $V$ is the coulomb interaction. In your derivation, you started off 
with the square of this matrix element, however, it is the two matrix elements 
involve different initial states that enter the expression for the cross 
sections. When both initial and final states
involve one free electron, it can be expanded as
\begin{equation}
V=2\sum_{t}\sum_{\tilde{j_i}j_0 \atop \tilde{j_f}j_1}\sum_{i\neq j}
  \left(Z^{t}(j_0,j_1)\cdot Z^{t}(\tilde{j_i},\tilde{j_f})\right)
  \Phi^{t}(j_0\tilde{j_i},j_1\tilde{j_f})
\end{equation}
\pagebreak
substitute into Eq. \ref{eq_cross0}
\begin{eqnarray}
\label{eq_cross1}
\sigma(J_fM_f,J_iM_i)&=&\frac{8\pi}{k_i^2}
  \sum_{\tilde{j_i},\tilde{j_i^\prime},\tilde{j_f} \atop m_{si},m_f}
  \sum_{j_0,j_1 \atop j_0^\prime,j_1^\prime}
  \sum_{t,t^\prime}
  (i)^{l_i-l_i^\prime}\exp[i(\delta_i-\delta_{i^\prime})]
  ([l_i][l_i^\prime])^\frac{1}{2} \nonumber \\
&&\times 
  (-1)^{j_i+j_i^\prime+2m_i}([j_i][j_i^\prime])^{\frac{1}{2}}
  \pmatrix{j_i & \frac{1}{2} & l_i \cr -m_i & m_{si} & 0 \cr}
  \pmatrix{j_i^\prime & \frac{1}{2} & l_i^\prime \cr -m_i & m_{si} & 0 \cr}
  \nonumber \\
&&\times \left<J_iM_i\tilde{j_i}m_i\left|
  Z^{t}(j_0j_1)\cdot Z^{t}(\tilde{j_i}\tilde{j_f})
  \right|J_fM_f\tilde{j_f}m_f\right> \nonumber \\
&&\times \left<J_iM_i\tilde{j_i^\prime}m_i\left|
  Z^{t^\prime}(j_0^\prime j_1^\prime)\cdot Z^{t^\prime}
  (\tilde{j_i^\prime}\tilde{j_f})
  \right|J_fM_f\tilde{j_f}m_f\right> \nonumber\\
&&\times \Phi^{t}(j_0\tilde{j_i},j_1\tilde{j_f})
  \Phi^{t^\prime}(j_0^\prime\tilde{j_i^\prime},j_1^\prime\tilde{j_f})
\end{eqnarray}
Expanding the scalar product in the spherical tensors
\begin{eqnarray}
\left<J_iM_i\tilde{j_i}m_i\left|
  Z^{t}(j_0j_1)\cdot Z^{t}(\tilde{j_i}\tilde{j_f})
  \right|J_fM_f\tilde{j_f}m_f\right> 
&=& \sum_{q}(-1)^{q}<\tilde{j_i}m_i|z^{t}_{-q}(\tilde{j_f}\tilde{j_f})
  |\tilde{j_i}m_i> \nonumber\\
& & \times <J_iM_i|Z^{t}_{q}(j_0j_1)|J_fM_f>  \nonumber\\
&=& (-1)^{q}(-1)^{j_i-m_i}
  \pmatrix{j_i & t & j_f \cr -m_i & -q & m_f \cr}
  \nonumber \\
& & \times (-1)^{J_i-M_i}
  \pmatrix{J_i & t & J_f \cr -M_i & q & M_f \cr}
  \nonumber \\
& & \times \left<J_iM_i\left|\left|Z^{t}(j_0j_1)\right|\right|J_fM_f\right>
\end{eqnarray}
The summation over $q$ may be dropped since it is fixed by the $3j$ 
symbols and is $M_i-M_f$, therefore
\begin{eqnarray}
\label{eq_cross2}
\sigma(J_fM_f,J_iM_i)&=&\frac{8\pi}{k_i^2}\sum_{tt^\prime}
  \pmatrix{J_i & t & J_f \cr -M_i & q & M_f \cr}
  \pmatrix{J_i & t^\prime & J_f \cr -M_i & q^\prime & M_f \cr}
  \nonumber \\
&&\times\sum_{j_0j_1 \atop j_0^\prime j_1^\prime}
  \left<J_iM_i\left|\left|Z^{t}(j_0j_1)\right|\right|J_fM_f\right>
  \left<J_iM_i\left|\left|Z^{t^\prime}(j_0^\prime j_1^\prime)
  \right|\right|J_fM_f\right> \nonumber\\
&&\times\sum_{\tilde{j_i},\tilde{j_i^\prime},\tilde{j_f} \atop m_{si},m_f}
  (i)^{l_i-l_i^\prime}\exp[i(\delta_i-\delta_{i^\prime})]
  ([l_i][l_i^\prime])^\frac{1}{2} \nonumber \\
&&\times
  ([j_i][j_i^\prime])^{\frac{1}{2}}
  \pmatrix{j_i & \frac{1}{2} & l_i \cr -m_i & m_{si} & 0 \cr}
  \pmatrix{j_i^\prime & \frac{1}{2} & l_i^\prime \cr -m_i & m_{si} & 0 \cr}
  \nonumber \\
&&\times
  \pmatrix{j_i & t & j_f \cr -m_i & -q & m_f \cr}
  \pmatrix{j_i^\prime & t^\prime & j_f \cr -m_i & -q^\prime & m_f \cr}
  \nonumber \\
&&\times \Phi^{t}(j_0\tilde{j_i},j_1\tilde{j_f})
  \Phi^{t^\prime}(j_0^\prime\tilde{j_i^\prime},j_1^\prime\tilde{j_f})
\end{eqnarray}
\pagebreak
or it may be written as
\begin{eqnarray}
\sigma(J_fM_f,J_iM_i)&=&\frac{8\pi}{k_i^2}\sum_{tt^\prime}
  \pmatrix{J_i & t & J_f \cr -M_i & q & M_f \cr}
  \pmatrix{J_i & t^\prime & J_f \cr -M_i & q^\prime & M_f \cr}
  \nonumber \\
&&\times\sum_{j_0j_1 \atop j_0^\prime j_1^\prime}
A^{t}(j_0j_1)A^{t^\prime}(j_0^\prime j_1^\prime)
Q^{tt^\prime}(j_0j_1,j_0^\prime j_1^\prime)
\end{eqnarray}
with
\begin{equation}
A^{t}(j_0j_1) =
\left<J_iM_i\left|\left|Z^{t}(j_0j_1)\right|\right|J_fM_f\right> 
\end{equation}
and
\begin{eqnarray}
Q^{tt^\prime}(j_0j_1,j_0^\prime j_1^\prime) &=&
  \sum_{\tilde{j_i},\tilde{j_i^\prime},\tilde{j_f} \atop m_{si},m_f}
  (i)^{l_i-l_i^\prime}\exp[i(\delta_i-\delta_{i^\prime})]
  ([l_i][l_i^\prime])^\frac{1}{2} \nonumber \\
&&\times
  ([j_i][j_i^\prime])^{\frac{1}{2}}
  \pmatrix{j_i & \frac{1}{2} & l_i \cr -m_i & m_{si} & 0 \cr}
  \pmatrix{j_i^\prime & \frac{1}{2} & l_i^\prime \cr -m_i & m_{si} & 0 \cr}
  \nonumber \\
&&\times 
  \pmatrix{j_i & t & j_f \cr -m_i & -q & m_f \cr}
  \pmatrix{j_i^\prime & t^\prime & j_f \cr -m_i & -q^\prime & m_f \cr}
  \nonumber \\
&&\times \Phi^{t}(j_0\tilde{j_i},j_1\tilde{j_f})
  \Phi^{t^\prime}(j_0^\prime\tilde{j_i^\prime},j_1^\prime\tilde{j_f})
\end{eqnarray}

In Eq. \ref{eq_cross2},
only after summation over $M_f$ and averaging over $M_i$, the first two 
$3j$ symbols produces $\delta_{tt^\prime}\delta_{qq^\prime}$, then
the summation over $m_f$ and $q$ in the last two $3j$ symbols gives
$\delta_{j_i j_i^\prime}$, and finally, the summation over $m_{si}$ and
$m_i$ gives $\delta_{l_i l_i^\prime}$,
therefore cancels the phase factors, we recover the HULLAC factorization 
for the total cross sections.













\section{Electron Impact Ionization}
\subsection{Introduction}
Several DW programs have been developed for the calculation of
electron impact ionization
(EII) cross
sections in the past decades, in either non-relativistic \cite{younger80} or
relativistic \cite{pindzola88, sampson91} approximations. The present program
is similar to that of \citet{sampson91} in particular,  which extends the
efficient factorization-interpolation method of \citet{barshalom:1988a} to the
calculation of EII cross sections, and leads to great simplifications for
complex ions. The essence of the factorization method is the separation of
the angular and radial parts in the formula of EII cross sections. The angular
part depends on the angular momentum coupling of the target states, while the
radial part only depends on the one-electron radial orbitals, without
explicit reference to the initial and final states involved in the
process. Different initial and final states affect the radial part only
implicitly through the ionization energy due to the energy conservation. Such
a separation enables the rapid calculation of the radial integrals on a few
ionization energy points, the integrals at actual ionization energies are
obtained by interpolation.  

The present program also implements much more efficient methods based on the
Coulomb-Born-exchange (CBE) approximation or the Binary-Encounter-Dipole (BED)
theory of \citet{kim94}. The CBE implementation makes use of the parameterized
reduced cross sections of \citet{golden:1977a, golden:1980a}, and the BED
implementation uses the bound-free differential oscillator strengths calculated
by the present program. The CBE results are close to the non-relativistic DW
cross sections, while the BED cross sections are substantially smaller at low
energies due to a scaling factor deliberately introduced into the theory, which
appear to improve the agreements with experimental results in many cases
\cite{kim94}.
 
\subsection{Relativistic Distorted-Wave Theory}
The formula for EII cross subsection, differential in energy of the ejected
electron, may be obtained from that for EIE by replacing one bound orbital in
the final state with the free orbital of the ejected electron and summing 
over its angular momentum. It can be expressed in terms of the collision
strength similar to that of EIE 
\begin{equation}
\label{eq_cross}
\sigma(\varepsilon_0,\varepsilon) = \frac{1}{k_0^2g_0}\Omega_{01},
\end{equation}
where $\varepsilon_0$ and $k_0$ are the energy and kinetic momentum of the
incident electron, $\varepsilon$ is the energy of the ejected electron. The
normalization of continuum orbitals is discussed in Sec.~\ref{sec:eie}. Note
that the absence of the 
factor $\pi$ as compared with the formula for the EIE cross subsection is due to
the different normalization of the free and bound orbitals. The collision
strength $\Omega_{01}$ is (see Sec.~\ref{sec:eie})
\begin{equation}
\label{eq_cs2}
\Omega_{01} = 2\sum_{\kappa,J_T\atop k,\alpha_0\beta_0}
Q^k(\alpha_0\kappa;\beta_0\kappa)
<\psi_0||Z^k(\alpha_0,\kappa)||\psi_1,\kappa;J_T>
<\psi_0||Z^k(\beta_0,\kappa)||\psi_1,\kappa;J_T>,
\end{equation}
where $\kappa$ is the relativistic angular quantum number of the ejected
electron, $J_T$ is the total angular momentum of the final state coupled with
the ejected electron,
the radial part $Q^k$ is identical to that for EIE, except that one of
the bound orbitals in the final state is now replaced by a free orbital. Note
that $Q^k$ contains the summation over the partial waves of incident and
scattered electrons.

As for photoionization (Sec.~{sec:pirr}), the angular factors involving the free
electron can be simplified using the decoupling formula of Racah
\begin{eqnarray}
&&\sum_{J_T}<\psi_0||Z^k(\alpha_0, \kappa)||\psi_1,\kappa;J_T>
<\psi_0||Z^k(\beta_0, \kappa)||\psi_1,\kappa;J_T> = \nonumber\\*
&(-1)^{j_{\alpha_0}-j_{\beta_0}}<&\psi_1||\tilde{a}_{\tilde{\alpha_0}}||\psi_0>
<\psi_1||\tilde{a}_{\tilde{\beta_0}}||\psi_0>
\sum_{J_T}[J_T]\sixj{J_1}{j}{J_T}{k}{J_0}{j_{\alpha_0}}
\sixj{J_1}{j}{J_T}{k}{J_0}{j_{\beta_0}},
\end{eqnarray}
where $[J_T]$ denotes $2J_T+1$ and $\sixj{j_1}{j_2}{j_3}{j_4}{j_5}{j_6}$ is
the Wigner $6j$ symbol. 
Substituting this into Eq.~(\ref{eq_cs2}), and carrying
out the summation over $J_T$ analytically, we find
\begin{equation}
\label{eq_scs}
\Omega_{01} = 2\sum_{k,\alpha_0\beta_0}
\delta_{j_{\alpha_0}j_{\beta_0}}[j_{\alpha_0}]^{-1} 
\overline{Q}^k(\alpha_0,\beta_0)
<\psi_1||\tilde{a}_{\tilde{\alpha_0}}||\psi_0>
<\psi_1||\tilde{a}_{\tilde{\beta_0}}||\psi_0>,
\end{equation}
where $\overline{Q}^k(\alpha_0,\beta_0) = \sum_\kappa
Q^k(\alpha_0\kappa;\beta_0\kappa)$.
Note that since only basis states with the same parity can mix, the condition
$j_{\alpha_0}=j_{\beta_0}$ implies that $l_{\alpha_0} = l_{\beta_0}$ as
well. Therefore, if the configuration interaction is limited within the same
$n$ complex, only terms with $\alpha_0 = \beta_0$ survive in the summation. 

The total ionization cross subsection is obtained by integrating $\Omega_{01}$
over the energy of the ejected electron, $\varepsilon$
\begin{equation}
\label{eq_total}
\sigma(\varepsilon_0) = \int\nolimits_0^{\frac{\varepsilon_0-I}{2}}
\sigma(\varepsilon_0,\varepsilon) d\varepsilon,
\end{equation}
where $I$ is the ionization energy. The ionization energy enters the
radial integral $Q^k$ implicitly, since $\varepsilon_0 =
I+\varepsilon_1+\varepsilon$, where $\varepsilon_1$ is the energy of the
scattered electron. We choose to calculate $Q^k$ as a function of $I$,
$\varepsilon_1$, and $\varepsilon$. For a given transition array, we first set
up a three dimensional grid on these three energies. The integrals $Q^k$ are 
only calculated on this grid, values for arbitrary energies are obtained by
interpolation. For a given ionization channel with
ionization energy of $I_{01}$ and the incident electron energy of
$\varepsilon_t + I_{01}$, where $\varepsilon_t$ is the sum of the energies of
the two final free electrons, the calculation of the total cross subsection
proceeds as 
following. The first step is the interpolation on the $I$ grid to obtain, for
each point on the $(\varepsilon_1,\varepsilon)$ grid, the values of
$\overline{Q}^k(I_{01})$ at the actual ionization energy, $I_{01}$. We then
establish another grid for the energy of the ejected electron in the range 
from 0 to $\varepsilon_t/2$ for the purpose of the numerical integration in
Eq.~(\ref{eq_total}). For each 
point on this grid, the value of $\varepsilon_1 = \varepsilon_t-\varepsilon$
is determined. A two dimensional interpolation is then performed on the array
$\overline{Q}^k(I_{01},\varepsilon_1,\varepsilon)$ to obtain the radial
integrals at these specific 
energies. Finally the integral in Eq.~(\ref{eq_total}) is evaluated to
yield the total cross subsection. In practice, we find such a procedure
minimizes the number of radial integrals need to be computed explicitly. The
number of grid points for the ionization energy is chosen similarly as for the
excitation energies of EIE, that is, 2 points with linear interpolation when
the ionization energies vary by less than 20\%, and 3 points with cubic spline
interpolation otherwise. The grids for the $\varepsilon_1$ and $\varepsilon$
are usually chosen to have 6 points spanning the range necessary to cover the
collision energies where the total cross sections are needed.

For each point on the three dimensional grid $(I,\varepsilon_1, \varepsilon)$,
and bound orbitals $\alpha_0$ and $\beta_0$, the evaluation of
$\overline{Q}^k(\alpha_0,\beta_0)$ involves the summation of the 
partial waves of all free electrons. The angular momentum of the ejected
electron must satisfy the triangular relation $\Delta(j_{\alpha_0}, k, j)$,
therefore, the limit on the rank $k$ in the expansion of the Coulomb
interaction restricts the number of $\kappa$ terms. Usually $k_{max} = 6$ is
large enough to ensure the convergence. A further limit on the orbital angular
momentum of 
the ejected electron may also be specified. The summation over the partial
waves of the incident and scattered electrons proceeds exactly the same way as
in the calculation of EIE cross sections, using the Coulomb-Bethe
approximation for the high partial waves in the case of allowed integrals.

\subsection{Coulomb-Born-Exchange Approximation}
The computation of radial integrals in the DW approximation is very time
consuming. In the CBE approximation implemented in the present program, only
the terms with $\alpha_0 = \beta_0$ are retained in Eq.~(\ref{eq_scs}). After
the integration over the energy of the ejected electron, a reduced radial
integral is defined as 
\begin{equation}
\label{eq_reduced}
Q_R(\alpha_0) = I_{\alpha_0}I
\sum_k\int_0^{\frac{\varepsilon_0-I}{2}}
\overline{Q}^k(\alpha_0,\alpha_0)d\varepsilon,
\end{equation}
where $I_{\alpha_0}$ is the binding energy of the orbital $\alpha_0$. It is well
known that the reduced radial integrals are not very sensitive to the difference
in the electronic structures and nuclear charges \cite{zhang90}. We therefore
use the hydrogenic values calculated under the CBE approximation
\cite{golden:1977a, golden:1980a}, in the place of detailed DW radial integrals.
A simple parameterized formula as given in \citet{zhang90} is used for this
purpose. 
\begin{equation}
Q_R = A\ln(u)+D\left(1-\frac{1}{u}\right)^2+
\left(\frac{c}{u}+\frac{d}{u^2}\right)\left(1-\frac{1}{u}\right),
\end{equation}
where $u=\varepsilon_0/I$. In fact, even in the detailed DW calculations, the
computed radial integrals are fitted with this formula, fixing the coefficient
$A$ according to the correct Bethe limit. The parameters are then used to
obtain total ionization cross sections at many incident energies. It is also
possible to use the parameters given by \citet{fontes93} instead of the CBE
values to bring the results in better agreement with the detailed DW
calculations. 

\subsection{Binary-Encounter-Dipole Theory}
The CBE approximation discussed above tries to capture the detail of the ionic
structure through a universal reduced radial integral. The BED theory combines
the Mott cross subsection and binary encounter cross subsection, taking into
account the high energy Bethe limit, to derive a semi-empirical formula for the
radial integrals \cite{kim94}. The computation of bound-free differential
oscillator strengths, which are related to the Bethe coefficients, is much
simpler than the DW calculation of ionization radial integrals. Therefore the
BED implementation is very efficient in practice. A unique feature of the BED
theory is the energy denominator $\varepsilon_0+I$ (for neutral atoms, the
average kinetic energy of the active bound electron should also be added), as
opposed to $\varepsilon_0$ in various first order Born approximations
(including DW). This modification effectively introduces a scaling factor
$\varepsilon_0/(\varepsilon_0+I)$, which reduces the ionization cross sections
at low energies. Comparison with experimental results for Low-Z near neutral
ions indicates that such scaling brings the theoretical results in better
agreement with experiments \cite{kim94}. However, for highly charged ions, we
find that this scaling reduces the cross sections by a too large factor. A
modified scaling factor is therefore introduced as
$\varepsilon_0/(\varepsilon_0+I_s)$ with
\begin{equation}
I_s = \frac{N}{2Z-N}I,
\end{equation}
where $N$ is the number of electrons of the ion and $Z$ is the atomic
number.


\section{Photoionization and Radiative Recombination}
\label{sec:pirr}
\subsection{Introduction}
The most accurate methods for calculating low energy
non-resonant photoionization (PI) and
recombination---both radiative recombination (RR) and dielectronic
recombination (DR) cross sections, are ones
based on the close-coupling (CC) approximation. The most widely used
implementation of the CC approximation is the 
R-matrix method \cite{hummer93, berrington95}. In this method, resonances are
treated in the most natural way through the coupling to closed channels. The
main drawback of the CC approximation is the prohibitively long computing time
required, especially at high energies \cite{zhang98}. The other class of
methods is based on the DW approximation, and ignores the inter-channel
coupling. The non-resonant background and the resonances are calculated
separately. In the present paper, we only discuss the non-resonant
PI and RR. Many calculations have been carried out using the DW approximation
employing different approaches to the bound and continuum wavefunctions
\cite{reilman79, clark86, verner93}. Most of previous theoretical results are
for PI from or RR onto a specific non-relativistic subshell, not for the
specific atomic states. \citet{zhang98} has extended their fully
relativistic code originally developed for the electron impact excitation
(EIE) to the 
calculation of PI and RR, and provided high energy PI cross sections to
complement the Opacity Project database \cite{seaton87}. The
present code is similar in technique to the program of \citet{zhang98}. To
improve the efficiency, we have extended the factorization-interpolation
procedure of \citet{barshalom:1988a} first developed for the calculation of EIE
cross sections to the calculation of PI and RR cross sections. 
In addition, we have implemented the option to calculate PI and RR cross
sections through multipole operators other than electric dipole (E1), although
no interference between different multipoles is taken into account. 

\subsection{Factorized Formula for Cross Sections}
The partial PI cross section can be expressed in terms of the differential
oscillator strength (in atomic units), and the partial RR cross section is
related to that of PI through the Milne relation
\begin{eqnarray}
\sigma_{PI} &=&
2\pi\alpha\frac{ d f}{ d E} \nonumber\\*
\sigma_{RR} &=& \frac{\alpha^2}{2}\frac{g_i}{g_f}
\frac{\omega^2}{\varepsilon(1+0.5\alpha^2\varepsilon)}
\sigma_{PI},
\end{eqnarray}
where $\alpha$ is the fine structure constant, $g_i$ and $g_f$ are the
statistical weight of the bound states before and after the photoionization
takes place respectively, $\omega$ is the photon energy, and $\varepsilon$ is
the energy of the ejected photo-electron. 
The differential oscillator
strength, $df/dE$, may be calculated similar to the bound-bound oscillator
strength through the generalized line strength
\begin{equation}
\frac{df}{dE} = \frac{\omega}{g_i}[L]^{-1}(\alpha\omega)^{2L-2}S,
\end{equation}
where $L$ is the rank of the multipole operator inducing the transition,
$[L]$ denotes $2L+1$, and the generalized line strength is 
\begin{equation}
\label{eq_S}
S = \sum_{\kappa J_T}\left|<\psi_f,\kappa;J_T||O^L||\psi_i>\right|^2,
\end{equation}
where $\kappa$ is the relativistic quantum number of the free electron, $J_T$
is the total angular momentum of the free state when the final bound state is
coupled to the continuum electron, and $O^L$ is the multiple operator
inducing the transition.

As in the calculation of bound-bound radiative transitions,
Eq.~(\ref{eq_S}) may be decomposed into an angular part and a radial part
(see Sec.~\ref{sec:structure})
\begin{equation}
\label{eq_Sf}
S = \sum_{\kappa J_T}\left|\sum_{\alpha\beta}
<\psi_f,\kappa;J_T||Z^L_M(\alpha,\beta)||\psi_i><\alpha||C^L||\beta>
M^L_{\alpha\beta}\right|^2,
\end{equation}
where $M^L_{\alpha\beta}$ is the one-electron radial integral and
$Z^L_M(\alpha,\beta)$ is the angular operator obtained by coupling the
creation and annihilation operators
\begin{equation}
Z^L_M(\alpha,\beta) = -[L]^{-1/2}\left[a^{\dagger}_{\hat{\alpha}}\times
\tilde{a}_{\hat{\beta}}\right],
\end{equation}
where the hat over the one electron state index indicates the inclusion of
magnetic quantum numbers $m$, and the tilde over the annihilation operator
denotes the transformation $\tilde{a}_{\hat{\beta}} = (-1)^{j_\beta -
m_\beta}a_{-\hat{\beta}}$ with $-\hat{\beta}$ being understood as the negation
of the magnetic quantum number $m_\beta$. 

Since the continuum orbital is absent in the initial bound state, the creation
operator $a^{\dagger}_{\tilde{\alpha}}$ must be chosen to be that of the free
electron. Using the decoupling formula of Racah, the reduced matrix elements
of the angular operator can be written as
\begin{equation}
<\psi_f,\kappa;J_T||Z^L_M(\kappa,\beta)||\psi_i>=(-1)^{J_T+J_i-L}[J_T]^{1/2}
<\psi_f||\tilde{a}_{\tilde{\beta}}||\psi_i>\sixj{J_f}{j}{J_T}{L}{J_i}{j_\beta},
\end{equation}
where $J_i$ and $J_f$ are the total angular momenta of the initial and final
states, $j$ is the angular momentum of the free electron, and
\sixj{j_1}{j_2}{j_3}{j_4}{j_5}{j_6} is the Wigner $6j$ symbol. Substituting
this into Eq.~(\ref{eq_Sf}) results in
\begin{eqnarray}
S=\sum_{\kappa\beta\beta^\prime}&\Bigg\{&\sum_{J_T}[J_T]
\sixj{J_f}{j}{J_T}{L}{J_i}{j_\beta}
\sixj{J_f}{j}{J_T}{L}{J_i}{j_{\beta^\prime}} 
\nonumber\\*
&\times&
<\psi_f||\tilde{a}_{\tilde{\beta}}||\psi_i>
<\psi_f||\tilde{a}_{\tilde{\beta^\prime}}||\psi_i> 
\nonumber\\*
&\times& 
<\kappa||C^L||\beta><\kappa||C^L||\beta^\prime>
M^L_{\kappa\beta}M^L_{\kappa\beta^\prime}\Bigg\}.
\end{eqnarray}
The summation over $J_T$ can be carried out analytically using the
orthogonality relation of $6j$ symbols to give
\begin{eqnarray}
S=\sum_{\kappa\beta\beta^\prime}&\Bigg\{&\delta_{j_\beta
j_{\beta^\prime}}[j_\beta]^{-1} <\psi_f||\tilde{a}_{\tilde{\beta}}||\psi_i>
<\psi_f||\tilde{a}_{\tilde{\beta^\prime}}||\psi_i>
\nonumber\\*
&\times&
<\kappa||C^L||\beta><\kappa||C^L||\beta^\prime>
M^L_{\kappa\beta}M^L_{\kappa\beta^\prime}\Bigg\}.
\end{eqnarray}
Note that the condition $j_\beta = j_{\beta^\prime}$ in the summation also
implies $l_\beta = l_{\beta^\prime}$, since only basis states with the same
parity can mix in $\psi_i$. 

The initial and final bound states are expressed as a sum over basis states
$\Phi_\mu$. The angular coefficients
$<\psi_f||\tilde{a}_{\tilde{\beta}}||\psi_i>$ are calculated as
\begin{equation}
<\psi_f||\tilde{a}_{\hat{\beta}}||\psi_i> = \sum_{\mu\nu}b_\mu b_\nu
<\Phi_\mu||\tilde{a}_{\hat{\beta}}||\Phi_\nu>,
\end{equation}
where $b_\mu$ are the mixing coefficients. The reduced matrix elements
$<\Phi_\mu||\tilde{a}_{\tilde{\beta}}||\Phi_\nu>$, which are related to the
coefficients of fractional parentage, are calculated using the method of
\citet{gaigalas97}. 

\subsection{Construction of Recombined Bound States}
In the calculation of PI and RR cross sections, bound states in two
neighboring charge 
states are involved. The central potential entering the Dirac equation may be
constructed with either the configurations of the ion before photoionization
(recombined ion) or after photoionization (recombining ion). The difference
between the two choices is small for high $Z$ elements. However, it can be
substantial for low $Z$ elements. As discussed in Sec.~\ref{sec:eie}, the
effective nuclear charge experienced by the free electron at large radial
distances is screened by one additional electron as compared to that
experienced by the bound electrons in the
recombining ion, and is the same as that experienced by the bound electrons in
the recombined ion. This suggests that
one should use the mean configuration of the recombined ion for the
construction of the central potential. 

The Hamiltonian matrix for the recombined ion may be constructed as for the
recombining ion by specifying appropriate configurations. However, it is
sometimes beneficial to use an alternative approach, which utilizes the already
diagonalized Hamiltonian of the recombining ion. In this method, the basis
states are constructed by coupling one additional electron to a specific
atomic state 
of the recombining ion $|\psi>$. However, if $|\psi>$ has a shell which is
equivalent to the additional electron, the antisymmetrization requires
non-trivial recoupling. Such recoupling is not needed when the additional
electron is high lying and interacts with the core recombining state
weakly. That is, the added electrons can be treated as spectators. We shall use
the alternative approach only in this case. In addition, only 
mixing of the basis states that have the spectator electron in the same
complex is 
allowed. When more general configuration interaction is needed, one should
construct the recombined states the same way as for recombining ones.
Since the core states already diagonalize the Hamiltonian of the
recombining ion, the addition of a weakly interacting electron should not
introduce large mixing of basis functions with a common core state. In fact,
when the 
principal quantum number of the spectator electron is large, only the mixing
between basis functions with the same core state need to be included. This
reduces the dimension of the Hamiltonian matrices substantially. 

The evaluation of the Hamiltonian matrix elements may also be simplified with
such basis functions. If the added electrons in the bra and ket states are
equivalent, the 
interacting electrons for both one and two-electron operators can all be from
the core states. Such terms only contribute to the diagonal elements, and are
simply the energy values of the core states. When one interacting electron can
be chosen to be the spectator, the kinetic and nuclear potential contributions
are also diagonal, which are their expectation values in the spectator
wavefunctions. With the aid of Racah algebra, the two-electron Coulomb
interactions in such cases may be written as \cite{barshalom:1988a} 
\begin{eqnarray}
<\psi_a,\alpha;J_TM_T|\sum_{i<j}\frac{1}{r_{ij}}|\psi_b,\gamma;J_TM_T>&=&
\sum_{k,\beta\delta}\Bigg\{(-1)^{j_\alpha+J_b+J_T}
\sixj{J_a}{j_\alpha}{J_T}{j_\gamma}{J_b}{k}
\nonumber\\*
&\times&
<\psi_a||Z^k(\beta,\delta)||\psi_b>P^k(\alpha\gamma;\beta\delta)\Bigg\},
\end{eqnarray}
where $\alpha$ and $\gamma$ are the spectator orbitals, $\beta$ and $\delta$
are the interacting electrons from the core states $\psi_a$ and $\psi_b$, $J_T$
and $M_T$ are the total angular momentum of the new basis function and its
projection, 
the radial integrals $P^k$ are identical to those involved in the DW collision
strength of EIE with free electrons now replaced with the spectators
(see Sec.~\ref{sec:eie}).

\subsection{Multipole Radial Integrals}
The radial integrals $M^L_{\kappa\beta}$ are the same as those involved in the
the calculation of bound-bound radiative transitions, with a free electron
replacing one of the bound orbitals. They may be calculated with the fully
relativistic expressions of \citet{grant74}. These expressions depend on
both the photon and the ejected photo-electron energies. However, when the
photon energy is not too high, the non-relativistic limits of the multipole
operators suffices. Using non-relativistic multipole operators to
evaluate radial integrals is the default of the program. 
Bound and continuum wavefunctions are obtained by solving the Dirac equations,
as discussed in Secs.~\ref{sec:structure} and \ref{sec:eie} in detail.

Unlike in bound-bound radiative transitions, the radial integrals now
depend on the photo-electron energy. In order to reduce the computing time,
when cross sections at a large number of collision energy points are needed,
the $M^L_{\kappa\beta}$ are calculated on an internal grid for the
photo-electron energy. The integrals needed at other energies are interpolated
on this grid. The grid points may be specified from the input, or if not
specified, it is generated according to the photo-electron energies where the
cross sections are to be calculated. For highly charged ions, where no Cooper
minimum is present, few points are needed for accurate interpolation. However,
when the PI cross sections show Cooper minima in the low energy region, which
often occurs in neutral or near neutral ions, a dense grid
should be used. On the other hand, the DW approximation at such low energies
in neutral or near neutral ions is rather inaccurate \cite{zhang98}.


\section{Autoionization and Dielectronic Recombination}
\subsection{Introduction}
The most accurate methods used today for treating resonances are ones based on
the close-coupling (CC) approximation, which yield unified results for both
background and resonant contributions. The efficient R-matrix implementation
of the CC approximation has been used extensively in the past to calculate
photoionization, recombination and electron impact excitation cross sections
\cite{hummer93, berrington95}. The main drawback of this approximation is
that it is very CPU-time consuming. Therefore, the number of coupled
channels are often limited in realistic calculations. Fortunately, in many
applications, especially for highly 
ionized ions, methods based on the distorted-wave (DW) approximation are
proven to be reasonably accurate even for resonant processes, e.g., as
indicated by the comparison between various DW calculations and experimental 
measurements of the dielectronic recombination (DR) rate coefficients for
O-like and F-like iron \cite{savin99}. In the DW methods, resonances are
usually treated as 
multi-step processes, one or more of these steps involve autoionization (AI)
or its inverse process, radiationless
electron capture. For example, the first step of DR is the radiationless
electron capture, followed by radiative decay, which competes with
autoionization. Therefore, the calculation of autoionization rates is the
major task in any DW treatments of resonant processes. Several DW programs
have been developed for this purpose in the past decades \cite{pindzola80,
kim87, oreg91}. The present method is most similar to that of
\citet{oreg91}, which uses an efficient factorization-interpolation procedure
for the calculation of the autoionization radial integrals. 

In the first order perturbation theory, the AI rate can be written as (in
atomic units) 
\begin{equation}
A^a = 2\sum_\kappa \left|
<\psi_f,\kappa;J_TM_T|\sum_{i<j}\frac{1}{r_{ij}}|\psi_i>\right|^2,
\end{equation}
where $\psi_i$ is the autoionizing state, $\psi_f$ is the final state which has
one less electron than $\psi_i$, $\kappa$ is the relativistic angular quantum
number of the free electron, whose wavefunction is normalized as discussed in
Sec.~\ref{sec:eie}. The total angular momentum of the coupled final state must
be equal to that of $\psi_i$, i.e.,  $J_T = J_i$ and $M_T = M_i$. After the
separation of the angular and radial integrals, we have
\begin{equation}
A^a = 2[J_i]^{-1}\sum_\kappa \left|\sum_{k,\alpha\gamma\delta}
<\psi_f,\kappa;J_T||Z^k(\alpha,\gamma)
\cdot Z^k(\kappa,\delta)||\psi_i>P^k(\kappa\delta;\alpha\gamma)\right|^2,
\end{equation}
where $\gamma$ and $\delta$ are the doubly excited bound orbitals in
$\psi_i$, $\alpha$ is the orbital which makes the internal transition in
$\psi_f$, and
$P^k$ is the radial integral as defined in the expression for the DW collision
strength of electron impact excitation, except that one of the free orbital
is replaced by a bound orbital (see Sec.~\ref{sec:eie}). For autoionization of
complex ions, the radial integrals $P^k$ with the same set of bound orbitals,
$\alpha$, $\gamma$, and $\delta$, 
often appear in many different transitions. However, the energies of the free
electrons in these integrals are different due to the conservation of
energy. It is 
computationally demanding to calculate such integrals for each transition
individually. Fortunately, the dependence of $P^k$ on the free electron energy
is rather weak, as noticed by \citet{oreg91}. Therefore, it is possible to
calculate $P^k$ at a few energies, and the integrals at actual transition
energies are obtained by interpolation from these values. Usually, for a given
class of 
transitions, a three-point grid spanning the entire range of transition
energies is sufficient.

The inverse process of AI is radiationless electron capture, or sometimes,
called dielectronic capture (DC). The cross sections for DC are related to AI
rates through the detailed balance. DC is a resonant process which only occur
at certain energy. These resonances are extremely narrow, and in most
applications, it is more appropriate to characterize each resonance by its
resonance strength, which is the cross section integrated over energy. In
atomic units, the DC strength can be written as
\begin{equation}
S_{DC} = \frac{g_i}{2g_f}\frac{\pi^2}{E_{if}}A^a,
\end{equation}
where $g_i$ and $g_f$ are the statistical weights of the autoionizing state
formed by DC and the target state before DC, respectively, and $E_{if}$ is the
resonance energy. 

The autoionizing state formed by DC may either autoionize, or radiatively
decay. Radiative decay to the states below the ionization limit completes the
recombination process, DR. Sometimes, the final state of decay lies above the
ionization limit, and may further decay to yield DR or autoionize. Therefore,
the radiative branching ratio for DR can be expressed as
\begin{equation}
\label{eq_B}
B(i) = \frac{\sum_k A^r(i\to k) + \sum_a A^r(i\to a)B(a)}
{\sum_{k^\prime}A^a(i\to k^\prime) + \sum_k A^r(i\to k) + \sum_a A^r(i\to a)},
\end{equation}
where $A^r$ represents radiative decay rates, $k$ denotes levels below the
ionization limit, $a$ denotes levels which 
may further autoionize, and $k^\prime$ are the final levels of
autoionization. In most cases, the effect of radiative decay to autoionizing
levels is small, an approximate expression for the branching ratio may be used
by dropping the final term in both the numerator and denominator of
Eq.~(\ref{eq_B}), that is 
\begin{equation}
\label{eq_BApprox}B(i) = \frac{\sum_k A^r(i\to k)}
{\sum_{k^\prime}A^a(i\to k^\prime) + \sum_k A^r(i\to k)}.
\end{equation}
This approximation turns out to be fairly good even if the decay to
autoionizing levels is not completely negligible as demonstrated and explained
by \citet{behar95, behar96}. The DR strength $S_{DR}$ is the product of DC
strength and the branching ratio.

In plasma modeling, the DR rate coefficients, $\alpha_{DR}$, for electrons in
the Maxwellian distribution is often needed, which can be expressed as
\begin{equation}
\label{eq_rate}
\alpha_{DR}(T) = \frac{h^3}{(2\pi m_e k_BT)^{3/2}}\sum_i\frac{g_i}{2g_f}
A^a(i\to f)B(i)\exp\left(-\frac{E_{if}}{k_BT}\right),
\end{equation}
where $T$ is the electron temperature, $m_e$ is the electron mass, $h$ is the
Planck constant, and $k_B$ is the Boltzmann constant. Since rate coefficients
are not fundamental atomic parameters, atomic units is not used in this
expression. 

The summation over the autoionizing levels $i$ in Eq.~(\ref{eq_rate})
extends to all possible states which may be formed by DC. Usually, the states
of each 
complex, designated as $nln^\prime l^\prime$, where the excited core electron
and the captured electron have the principal quantum numbers, $n$ and
$n^\prime$, are considered separately. 
The number of different $nl$ terms for the excited core
electron need to be considered is not large. Usually, the $\Delta n = 0$
resonances, where the transition of the core electron does not change its
principal quantum number, and $\Delta n = 1$ resonances, where the principal
quantum number of the core electron changes by one unit, make
dominant contributions to the 
total DR rate coefficients at temperatures of interest. The summation
over $n^\prime l^\prime$, however, can potentially involve large number of
terms. For high $n^\prime$, the contributions from large $l^\prime$ are
negligible. The maximum $l^\prime$ of 6--8 and $\sim 12$ are sufficient for
$\Delta n \ne 0$ and $\Delta n = 0$ resonances, respectively. The maximum
$n^\prime$ for $\Delta n \ne 0$ resonances is usually chosen to be $\sim 15$,
beyond which, the $n^{\prime -3}$ scaling law may be used for
extrapolation. For $\Delta n = 0$ resonances, the minimum $n^\prime$ which
opens some low energy channels can be large, and the $n^{\prime -3}$ scaling
law is only valid for very large $n^\prime$. Nevertheless, the partial DR rate
coefficients as a function of $n^\prime$ is quite smooth except for some
irregularities. Therefore, detailed 
calculations can be carried out on a suitable $n^\prime$ grid.
Interpolation may be used to complete the summation. However, one should pay
special attention to those $n^\prime$ where a specific channel starts to
open, because the rate coefficients is discontinuous at such $n^\prime$. In
addition, the rate coefficients of very low energy resonances are quite
sensitive to the exact value of the resonance energies. The theoretical
energies should be used with caution for resonances below a few tens of
eV. Sometimes, shifting the resonance energies for an entire series according
to the experimental transition energies of the core prove to be practical.
