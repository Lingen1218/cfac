\chapter{Frequently Asked Questions (FAQ)}
\section{General}
\faq{What operating systems does \cFAC run on}{
\cFAC is written in a mixture of ANSI C and Fortran 77, all of them are in
principle platform independent. However, the mixed language programming makes it
more easily installed in modern UNIX-like systems than others. So far, it has
been tested to work under Solaris, Linux, Mac OS X, and MS Windows with the UNIX
API emulation provided by Cygwin.}
  
\faq{How does \cFAC differ from other atomic codes}{
The theoretical methods used in \cFAC are similar to some other distorted-wave
atomic codes, such as HULLAC and differ from more elaborate programs based on
close-coupling approximations, such as the Belfast R-Matrix code. The biggest
advantage of \cFAC is its ease of use and its scriptability.}

\faq{Can \cFAC be incorporated into XSPEC or other spectral analysis programs}{
In principle, the collisional radiative model comes with \cFAC can be used in
XSPEC or similar spectral analysis programs. However, this is not practical,
as the model includes a large number of atomic states, especially, the doubly
excited states to treat the resonant processes, and is therefore very time
consuming. The best strategy of incorporating the \cFAC results to external
spectral models is to extract basic atomic or plasma parameters and use them
to build table models or implement dedicated subroutines.}

\faq{Which atomic processes can or cannot be calculated with \cFAC}{
\cFAC can calculate energy levels, radiative transition rates of arbitrary
multipole type, collisional excitation and ionization cross sections by
electron impact, photoionization and radiative recombination cross sections and
autoionization rates. In the current form, \cFAC does not treat two-photon
decay, although such decay rates of $2s S_{1/2}$ state of H-like ions and
$1s2s S_{0}$ state of He-like ions are included in the collisional radiative
model using interpolation formulae taken from literature. The three-body
recombination are not implemented as well.}

\faq{What is a typical accuracy of the atomic parameters calculated with
\cFAC}{
The ions other than H-like, the accuracy of energy levels are usually a few
eV, which translates to 10--30 m{\AA} for the wavelength at $\sim$10{\AA}. For
radiative transition rates and cross sections, the accuracy is
$\sim$10--20\%. Data for near-neutral ions or atoms may have even larger
errors.}

\faq{Can \cFAC be used to calculate atomic parameters for non-X-ray (UV,
optical, etc.) lines}{
For multiply-charged ions, non-X-ray lines usually result from the transitions
within the same complex, which usually have large relative uncertainties in
the calculated wavelengths and transition rates. The accuracy for UV and
optical lines from near-neutral ions is also very limited.}

\faq{Are there standard references to \cFAC}{
Currently, no papers have been published describing the code, though drafts have
been written, which are included in the \cFAC distribution. M. F. Gu was trying
to find a suitable journal willing to publish them, originally submitting to
Computer Physics Communications (CPC). However, the referee and editor
complained about the lack of documentation and code comments. The documentation
has improved since then, the code commenting still needs extensive work, which
may not be done soon. The first paper that used results of FAC is Gu, M.F.,
2003, ApJ, 582, 1241, which can be used as the reference.}

\faq{How do I report bugs, make suggestions and get updated about new
versions}{
Please contact the author at \textbf{mfgu@space.mit.edu} for bugs and
suggestions. I maintain a small email address list of people who expressed
interest in \cFAC and send release announcements to them. Let me know if you want
to be added to this list. If the list ever grows to the point when I can no
longer put it under my personal address book, we may have to create a
dedicated mailing list.}

\section{Atomic Structure}
\faq{How do I specify a bare ion}{
The bare ion is indicated by a call to \key{Config('', group='b')}, i.e.,
the first argument of \key{Config} is an empty string.}

\faq{Which configurations should be used as basis for the mean configuration
which optimizes the radial potential}{
The function \key{OptimizeRadial} accepts a list of configurations, which
form the basis for the construction of the mean configuration. Usually, only
the lowest lying configurations should be used in \key{OptimizeRadial} for
the construction of mean configuration. I have found that using configurations
corresponding to the ground complex is always a good idea. Sometimes, it maybe
worthwhile to include the first excited configurations. However, it is always a
bad practice to include very highly excited configurations, especially those
inner shell excited ones.}

\faq{Can I use a specific mean configuration to be used in potential
optimization}{
The function \key{AvgConfig} may be used to set the mean configuration for
the potential optimization. In this case, the function
\key{OptimizeRadial} must be called with no arguments.}

\faq{How do I know what mean configuration is used in the potential
optimization, If one is not given specifically by \key{AvgConfig}}{
The function \key{GetPotential} may be called after \key{OptimizeRadial}
to obtain the mean configuration used, and the resulting radial potential.}

\faq{When calculating ionization or recombination processes, should I use the
mean configuration for recombined (ionizing) or recombining (ionized) ion}{
Usually, it does not matter for highly charged ions, and I usually use the 
recombined ion. The difference of one more or less electron screening the
nuclear charge may be substantial for low-$Z$ elements, in which case, one
may have to make a decision by comparing the results to experimental values or
other theoretical works.}

\faq{How do I determine which configurations should be interacting}{
This depends on the computer resource available, and the desired accuracy. The
dimension of the Hamiltonian matrix increases very rapidly as the number of
interacting configurations grows. The convergence with respect to the
configuration interaction is usually slow. For applications which \cFAC is
primarily designed for, one typically includes only configurations within the
same complex, except for some low-lying configurations.}

\faq{What relativistic effects are included}{
The standard Dirac-Coulomb Hamiltonian is used in \cFAC, which means that the
spin-orbit interaction, mass-effect and other leading relativistic effects are
fully treated. However, higher-order QED effects, such as retardation and
recoil are only included in the Breit interaction with zero energy limit for
the exchanged photon. Vacuum polarization and self-energy corrections are
treated in the screened hydrogenic approximation.}

\faq{Why the transition rate between two specific states is not in the output
file, although it should have been calculated}{
Not every calculated transition rate are output. Some weak transitions are
discarded to avoid very large files. A small number, which may be set by the
function \key{SetTransitionCut}, controls this behavior. If the
transition rate of $2\to 1$ divided by the total decay rate of state 2 is less
than this number, this rate is not output. The default for this number is
$10^{-4}$.}

\section{Collisional Excitation}
\faq{Why is there multiple data blocks in the output corresponding to a single
call of \key{CETable} sometimes}{
Sometimes, the transition array corresponding to a given call to \key{CETable}
include transitions with a wide range of excitation energies. This typically
happens for transitions within a single complex or transitions between more
than 2 complexes are mixed together in one call of \key{CETable} (which should
generally be avoided). Since the excitation radial integrals are only
calculated on a few-point transition energy grid, it is undesirable to have a
very wide range in the actual transition energies. \key{CETable} avoid this by
subdivide the transitions in groups. Within each group, the transition
energies does not vary by more than a factor of 5. A different transition
energy grid and collision energy grid are used for different groups, and
therefore, corresponding to different data blocks in the output.}

\faq{Why is the default \key{QKMODE} for excitation is \key{EXACT}, not
\key{FIT}}{
It used to be in \key{FIT} mode. However, the fitting formulae sometimes fail
to reproduce the calculated collision strengths. After playing with different
fitting formulae for a while, I decided that the user should do the fitting
(if one is desired) on the case by case basis. The function \key{SetCEQkMode}
may be used to specify a different \key{QKMODE}.}

\faq{Can I use a different collision energy grid}{
A collision energy grid in terms of the energy of the scattered electron is
automatically constructed if one is not specified prior to calling
\key{CETable}. One may use the function \key{SetCEGrid} to specify a different
grid. Or one may use \key{SetUsrCEGrid} to have a user grid different from the
grid on which the collision strengths are calculated, and use
\key{INTERPOLATE} mode for the \key{QKMODE}.}

\faq{The collision strengths at very high energies are incorrect}{
Due to the limited radial grid size, the collision strengths at energies much
higher than the excitation energy (more than a few hundred times higher) are
unlikely to be reliable. However, the high energy collision strengths should
not be calculated directly. The Bethe and Born limit parameters in the output
should be used to obtain them.}

\section{Photoionization and Radiative Recombination}
\faq{When can I use the function \key{RecStates} to construct the
recombined states instead of specifying their configurations by \key{Config}}{
The function \key{RecStates} can only be used if the free electron is captured
to an empty orbital.}

\faq{Why is the bound-free oscillator strength differ from some other
theoretical calculation by a constant factor}{
The bound-free differential oscillator strengths calculated by \cFAC have units
of Hartree$^{-1}$. Also the values depend on the normalization of continuum
orbitals. It is therefore possible that they differ from other theoretical
calculations by a constant factor. One should always use the formula in this
manual or the accompanying papers to convert them to photoionization or
radiative recombination cross sections.}

\faq{Can I use only the fitting formula and ignore the tabulated $gf$
values}{
The fitting formula and the parameters given for the bound-free oscillator
strengths is only valid at high energies (beyond the largest energy of the
photo-electron energy grid). This means that the $gf$ values at energies
within the photo-electron energy grid should be calculated by interpolation
instead of the fitting formula. However, this is often only necessary for the
ionization of valence shells of near neutral ions and atoms, since only in
these cases, the near threshold behavior of the $gf$ values differ from the
fitting formula significantly.}

\section{Autoionization and Dielectronic Recombination}{
\faq{What does the channel number in the call to \key{AITable} mean}{
It does not mean anything. It is simply an identifier which may be useful
sometimes to tag a certain autoionization channel.}

\faq{How do I improve the resonance energies of low-lying $\Delta n = 0$
resonances}{
The energies of low-lying $\Delta n = 0$ resonances can not the calculated
accurately. This greatly reduces the reliability of the resulting dielectronic
recombination rates. In \cFAC, there is a method to improve the accuracy of
these energies by adjusting the core transition energies according to the
experimental values. The function \key{CorrectEnergy} is used to modify the
calculated energy levels by specified amount.}

\faq{Why is it advised to make separate calls to \key{AITable} for different
bound or free state complexes}{
The autoionization radial integrals are only calculated for a few free
electron energies. The actual values are interpolated from them. This only
works well if the the free electron energy after autoionization does not vary
widely. Therefore, one should avoid calling \key{AITable} with bound or free
states in different complexes. e.g., KLL and KLM resonances should be
calculated with two separate calls to \key{AITable}.}

\section{Collisional Ionization}
\faq{Why are \cFAC ionization cross sections calculated with \key{BED} mode
usually much smaller at near threshold energies as compared with
distorted-wave calculations}{
When using \key{BED} mode to calculate the ionization radial integrals, the
total ionization cross sections are scaled by a factor $E/(E+I)$, where $E$ is
the energy of incident electron, and $I$ is the ionization threshold
energy. The result of this scaling is usually desirable as distorted-wave
method overestimates the near threshold cross sections.}

\faq{Why is the \key{DW} mode so slow as compared with \key{BED} and \key{CB}
modes}{
In the \key{DW} mode, the ionization radial integrals are calculated by
summing up partial wave contributions. This is a time consuming process as
now there are two continuum electrons involved. In the \key{CB} mode, the
radial integrals are simply looked up in a table, is therefore the fastest
method. In the \key{BED} mode, the radial integrals are calculated using the
bound-free differential oscillator strengths, which can be computed much
faster than the \key{DW} ionization radial integrals.}
